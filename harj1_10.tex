    \begin{harj}\label{rajaArvo}
        Tuota seuraava dokumenttiisi:
        \begin{sample}
            Funktion \(f\colon \R\to \R\) derivaatta pisteessä \(x_0\in\R\) on
            \[
                f'(x_0)=\lim_{x\to x_0}\frac{f(x)-f(x_0)}{x-x_0},
            \]
            mikäli raja-arvo on olemassa.
        \end{sample}
        \begin{itemize}
            \item Kaksoispisteen paikalla kannattaa käyttää komentoa \lstinline-\colon-
            \item Symbolit \(\mathbb{R}\), \(\to\), \(\in\) ja \(\lim\) saa komennoilla \lstinline-\mathbb{R}-, \lstinline-\to-, \lstinline-\in- ja \lstinline-\lim-.
            \item Derivointipilkku tulee heittomerkillä \lstinline-'- .
            \item Alaindeksin (symbolille tai operaattorille) saa kirjoittamalla \lstinline-symboli_{alaindeksi}-.
        \end{itemize}
    \end{harj}
