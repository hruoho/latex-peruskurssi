    \begin{harj}
        Kirjoita seuraavanlainen matriisitoimitus: 
        \begin{align*}
            &
            \begin{bmatrix*}[r]
                0 & 0 & 1 & a_1\\
                0 & 1 & 1 & a_2\\
                1 & 1 & 1 & a_3
            \end{bmatrix*}
%        \stackrel{\begin{scriptsize} R_1\leftrightarrow R_3 \end{scriptsize} }{\longrightarrow}
        \xrightarrow{R_1\leftrightarrow R_3}
            \begin{bmatrix*}[r]
                1 & 1 & 1 & a_3\\
                0 & 1 & 1 & a_2\\
                0 & 0 & 1 & a_1
            \end{bmatrix*}
            %\\
            %\stackrel{\begin{scriptsize} R_1-R_2 \end{scriptsize} }{\longrightarrow}
            %&\begin{bmatrix*}[r]
            %1 & 0 & 0 & a_3-a_2\\
            %0 & 1 & 1 & a_2\\
            %0 & 0 & 1 & a_1
            %\end{bmatrix*}
            %\stackrel{\begin{scriptsize} R_2-R_3 \end{scriptsize} }{\longrightarrow}
            %\begin{bmatrix*}[r]
            %1 & 0 & 0 & a_3-a_2\\
            %0 & 1 & 0 & a_2-a_1\\
            %0 & 0 & 1 & a_1
            %\end{bmatrix*}
        \end{align*}
        Matriisien sisällön saat päättää vapaasti eikä rivitoimituksen tarvitse mennä oikein. % Matriisien välisen merkinnän saat rakennettua komennon \verb-\stackrel{}{}- avulla. Yllä on käytetty komentoa
        Kaavan mukana automaattisesti venyvän nuolen saa komennolla \verb-\xrightarrow{kaava}-.
%        \begin{verbatim}
%        \stackrel{
%            \begin{scriptsize} 
%                R_1\leftrightarrow R_3 
%            \end{scriptsize}
%        }{\longrightarrow} 
%        \end{verbatim}

        % Kannattaa kopioida alkuperäisen matriisin koodi ja muokata sitä kunkin uuden matriisin kohdalla. Matriisit vievät paljon tilaa, joten ympäristö \verb-align*- on hyödyksi!
    \end{harj}
