\documentclass[handout,%
hyperref={unicode,colorlinks=true}]{beamer}
\usecolortheme{seahorse}
\usepackage{etex}   % lisää rekistereitä
%\usefonttheme{serif}
\usefonttheme[onlymath]{serif}     % serif vain matikkamoodiin
\usepackage[utf8]{inputenc}
\usepackage[T1]{fontenc}
\usepackage[finnish]{babel}
\usepackage{amsmath}
%\usepackage{amsfonts}
\usepackage{amssymb}
\usepackage{tikz}
\usepackage{framed}
\usepackage{alltt}
\usepackage{fancyvrb}
\usepackage{multirow}
\usepackage{mathtools}
\usepackage{multicol}
\usetikzlibrary{matrix,arrows,decorations.pathmorphing}
\usetikzlibrary{fadings,patterns,arrows}
\pgfmathsetmacro{\inf}{-2}
\pgfmathsetmacro{\sup}{2}
\tikzstyle{MyPlotStyle}=[domain=\inf:\sup,samples=100,smooth]
\usepgflibrary{arrows}

\usepackage{float}

\usepackage{lmodern}
\usepackage{microtype}
\usepackage{listings}
\lstset{
    language=[LaTeX]TeX,
    basicstyle=\ttfamily,%\footnotesize,
    commentstyle=\ttfamily,%\footnotesize,
    columns=fullflexible,
    keepspaces=true,
    backgroundcolor=\color{gray!20},
    escapechar=<>,
    morekeywords={\part,\chapter,\subsection,\subsubsection,\paragraph,\subparagraph,%
                  \middle,\theoremstyle,\eqref,\subsetneq},
    literate={ä}{{\"a}}1
             {Ä}{{\"A}}1
             {ö}{{\"o}}1
             {Ö}{{\"O}}1
}
\lstdefinelanguage{BibTeX}
    {keywords={%
        @article,@book,@collectedbook,@conference,@electronic,@ieeetranbstctl,%
        @inbook,@incollectedbook,@incollection,@injournal,@inproceedings,%
        @manual,@mastersthesis,@misc,@patent,@periodical,@phdthesis,@preamble,%
        @proceedings,@standard,@string,@techreport,@unpublished%
    },
    comment=[l][\itshape]{@comment},
    sensitive=false,
}

\newcommand{\cns}[1]{\texttt{#1}}   % vakioille yms.
\newcommand{\newterm}[1]{\textit{#1}}

\newcounter{harkka}
\newtheorem{lause}{Lause}
\newtheorem*{huom}{Huomautus}
\newcounter{laskuri}
\newtheorem{trma}{Teoreema}[laskuri]
\newtheorem{prop}[trma]{Propositio}
\theoremstyle{remark}
\newtheorem{harj}{Harjoitus}[section]

\newenvironment{fframe}
    {\begin{frame}[fragile,environment=fframe]}
    {\end{frame}}

\newtheorem*{esim}{Esimerkki}
\newtheorem*{ratk}{Ratkaisuehdotus}
\newtheorem*{thmextra}{Lisätieto}
\newenvironment{extra}{\begin{thmextra}\footnotesize}{\end{thmextra}}
\newenvironment{serif}{\fontfamily{lmr}\selectfont}{}
\usepackage{ragged2e}   % \justifying
\newenvironment{sample}{\begin{framed}\justifying\begin{serif}}{\end{serif}\end{framed}}



%Matriisien säätö:
\makeatletter
\renewcommand*\env@matrix[1][*\c@MaxMatrixCols c]{%
    \hskip -\arraycolsep
    \let\@ifnextchar\new@ifnextchar
\array{#1}}
\makeatother

\newcommand{\numeroi}{\stepcounter{harkka}\arabic{harkka}}
\newcommand{\vaihto}{\\[1em]}
\newcommand{\R}{\mathbb{R}\,}
\newcommand{\N}{\mathbb{N}\,}
\newcommand{\Z}{\mathbb{Z}\,}
\newcommand{\Q}{\mathbb{Q}\,}
\providecommand{\C}{}               % XeTeX:illä tämä on jokin aksenttikomento
\renewcommand{\C}{\mathbb{C}\,}
\newcommand{\abs}[1]{\left|#1\right|}
\newcommand{\va}{\bar{a}}
\newcommand{\vb}{\bar{b}}
\newcommand{\vw}{\bar{w}}
\newcommand{\vv}{\bar{v}}
%\newcommand{\abs}[1]{\left|#1\right|}
%\newcommand{\BibTeX}{\textsc{Bib}\negthinspace\TeX}
\newcommand{\BibTeX}{Bib\negthinspace\TeX}  % textsc toimii lmodernilla, mutta näyttää pahalta...
\newcommand{\TikZ}{Ti\textit{k}Z}
%\newcommand{\bino}[2]{(x-#1)^{#2}}
\newcommand{\bino}[3]{\binom{#1}{#3}{#2}^{#3}\left(1-#2\right)^{#1-#3}}
\setbeamertemplate{theorems}[numbered]
\newcommand{\set}[1]{ \{ #1 \} }

\title{\LaTeX}



% Mitä sisällytetään
\includeonly{
1kerta
}

\begin{document}

\begin{fframe}
    \begin{harj}
        Tuota seuraava lause dokumenttiisi:
        \begin{sample}
            Vaihdannaisessa renkaassa pätee \((a+b)^2=a^2+2ab+b^2\). Jos siis \(2ab\neq0\), niin \((a+b)^2\neq a^2+b^2\). 
        \end{sample}
        Erisuuruuden saat komennolla \lstinline-\neq-. Eksponentti kirjoitetaan tyyliin \lstinline-x^2- (tulostaa \(x^2\)).
    \end{harj}
\end{fframe}

\begin{fframe}
    \begin{harj}
        \label{rajaArvo}
        Tuota seuraava dokumenttiisi:
        \begin{sample}
            Funktion \(f\colon \R\to \R\) derivaatta pisteessä \(x_0\in\R\) on
            \[
                f'(x_0)=\lim_{x\to x_0}\frac{f(x)-f(x_0)}{x-x_0},
            \]
            mikäli raja-arvo on olemassa.
        \end{sample}
        \begin{itemize}
            \item Kaksoispisteen paikalla kannattaa käyttää komentoa \lstinline-\colon-
            \item Symbolit \(\mathbb{R}\), \(\to\), \(\in\) ja \(\lim\) saa komennoilla \lstinline-\mathbb{R}-, \lstinline-\to-, \lstinline-\in- ja \lstinline-\lim-.
            \item Derivointipilkku tulee heittomerkillä \lstinline-'- .
            \item Alaindeksin (symbolille tai operaattorille) saa kirjoittamalla \lstinline-symboli_{alaindeksi}-.
        \end{itemize}
    \end{harj}
\end{fframe}

\begin{fframe}
    \begin{harj}
        Tuota seuraava dokumenttiisi:
        \begin{sample}
            Jos \(F\) on \(\sigma\)-algebra ja \(A_i\in F\) kaikilla \(i= 1,2,\dots\), niin 
            \[
                \bigcup_{i=1}^{\infty} A_i\in F.
            \]
        \end{sample}
        Tarvitset mm. komentoja \lstinline-\sigma-, \lstinline-\in-, \lstinline-\bigcup- ja \lstinline-\infty-. Ala- ja yläindeksit kirjoitetaan merkkien \lstinline-_- ja \lstinline-^- avulla (jos indeksiin halutaan enemmän kuin yksi merkki, se täytyy laittaa aaltosulkeisiin kuten edellisessä tehtävässä). 
    \end{harj}
\end{fframe}

\begin{fframe}
    Yleisimpiä matemaattisia symboleita löytää Texmakerin vasempaan laitaan avautuvista valikoista. Muuten niitä voi etsiä esim. seuraavista osoitteista:
    \begin{scriptsize}
        \begin{itemize}
            \item \url{http://www.tex.ac.uk/tex-archive/info/symbols/comprehensive/symbols-a4.pdf}
            \item \url{http://detexify.kirelabs.org/classify.html}
        \end{itemize}
    \end{scriptsize}
    \begin{harj}
        Selvitä, miten voit tuottaa vektorimerkinnät \(\bar{v}\), \(\bar{w}\) ja \(\overline{AB}\). Kirjoita seuraava:
        \begin{sample}
            Vektoreiden \(\bar{v}=\overline{AB}\) ja \(\bar{w}=\overline{CD}\) ristitulo \(\bar{v}\times\bar{w}\) on kohtisuorassa kumpaakin vektoria vastaan. Vektoreiden pistetulo \(\bar{v}\cdot\bar{w}\) on sen sijaan reaaliluku.
        \end{sample}
    \end{harj}

\end{fframe}

\begin{fframe}
    \begin{harj}
        Tuota seuraava dokumenttiisi:
        \begin{sample}
            Todista seuraava yhtälö:
            \[
                \left\lbrace 2^n \,\middle|\, n\in\Z\right\rbrace=\left\lbrace \left(\frac{1}{2}\right)^{-n}\,\middle|\, n\in\Z\right\rbrace
            \]
        \end{sample}
        Komennolla \lstinline-\middle|- saat pystyviivan automaattisesti oikean kokoisena ja komennolla \lstinline-\,- hieman ylimääräistä väliä niiden ympärille. Tehtävästä \ref{rajaArvo} saat apua symboleiden \(\in\) ja \(\Z\) luomiseen.
    \end{harj}
\end{fframe}

\begin{fframe}
    \begin{harj}
        Kirjoita seuraava:
        \begin{sample}
            Jos \(F\) on \(\sigma\)-algebra ja \(P\colon F\to\R\) todennäköisyys, niin tapahtumille \(A_1,A_2,\dotsc\in F\) pätee
            \[
                P\left(\bigcup_{i=1}^\infty A_i\right) \leq \sum_{i=1}^\infty P(A_i).
            \]
        \end{sample}
    \end{harj}
\end{fframe}

\begin{fframe}
    \begin{harj}
        \label{viittausTehtava}
        Tuota seuraava dokumenttiisi:
        \begin{sample}
            Eksponenttifunktiolla \(e^x\) on sarjakehitelmä
            \begin{equation*}
                \hskip\textwidth minus \textwidth e^x=\sum_{n=1}^\infty\frac{x^n}{n!}.\hskip\textwidth minus \textwidth (1)
            \end{equation*}
        \end{sample}
        Summamerkinnän saat komennolla \lstinline-\sum_{}^{}-, osamäärän komennolla \lstinline-\frac{}{}- ja äärettömän symbolin komennolla \lstinline-\infty-. 
    \end{harj}
\end{fframe}

\end{document}