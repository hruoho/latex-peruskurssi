\section{4.kerta}


\subsection{Viittaaminen kuviin ja taulukoihin}

\begin{fframe}
    \frametitle{Viittaaminen kuviin ja taulukoihin}
    Kelluviksi kirjoitetut kuvat ja taulukot tulevat numeroiduiksi (tarkemmin sanottuna niiden otsikot kuten kuvateksti), jolloin niihin viittaaminen onnistuu helposti komennoilla \lstinline-\label{}- ja \lstinline-\ref{}-. 
    %\begin{itemize}
    %\item \verb-\ref{nimi}- (tulostaa viitattavan kohteen numeron)
    %\item \verb-\eqref{nimi}- (tulostaa kohteen numeron sulkujen sisällä)
    %\item \verb-\pageref{nimi}- (tulostaa sen sivun numeron, jolla kohde on)  
    %\end{itemize}
    %Sisäisten viittausten kanssa tulee aina käyttää komentoja. Tällöin viittaukset pysyvät kohdallaan vaikka numeroinnit muuttuisivat työn edetessä.
    %\end{fframe}
    %\begin{fframe}
    %Huom! Uuden viittauksen jälkeen työn joutuu ajamaan kahdesti, jotta numeroinnit tulevat näkyviin (kahden kysymysmerkin sijaan).
    \begin{harj}
        Viittaa harjoituksessa \ref{kelluvaTaulukko} luomaasi kelluvaan taulukkoon. Käytä komentoja \lstinline-\ref{}- ja \lstinline-\pageref{}-. Komento \lstinline-\label{..}- tulee otsikon luovan \lstinline-\caption{..}--komennon jälkeen.
        %\vaihto
        %Muista, että viitattavalle kohteelle on ensin annettava tunnus komennolla \verb-\label{valitsemasi tunnus}-. Komento sijoitetaan ympäristön aloittavan komennon \verb-\begin{}- perään.
    \end{harj}
    \begin{harj}
        Sama kuin edellinen harjoitus, mutta viittaa harjoituksessa \ref{kelluvaTikz} luomaasi kelluvaan kuvaan.
    \end{harj}
\end{fframe}


\subsection{Kirjallisuusviitteet}

\subsubsection{Lainaukset ja alaviitteet}
\begin{fframe}
    \frametitle{Lainaukset ja alaviitteet}
    \begin{itemize}[<+->]
        \item Suomalaisittain käytettävät lainausmerkit saa kahdella peräkkäisellä \lstinline-'- merkillä:
            \begin{lstlisting}
Matti sanoi: ''Onpa outoa!''<>
            \end{lstlisting}
            tulostaa
            \begin{sample}
                Matti sanoi: ''Onpa outoa!''
            \end{sample}
        \item Lainauksia varten on olemassa myös omia ympäristöjään, kuten \cns{quote} ja \cns{quotation}.
        \item Alaviitteen voi luoda komennolla \lstinline-\footnote{teksti}-. Komento kirjoitetaan siihen kohtaan, johon alaviitteen merkki halutaan. 
    \end{itemize}
\end{fframe}

\begin{fframe}
    \frametitle{Lukuvälit ja ajatusviiva}
    \begin{itemize}[<+->]
        \item Lukuväleissä oleva viiva kirjoitetaan \lstinline|--|, esim.
            \begin{lstlisting}
Vertaa s.~39--42 ja s.~39-42.<>
            \end{lstlisting}
            tulostaa
            \begin{sample}
                Vertaa s.~39--42 ja s.~39-42.
            \end{sample}
        \item Tällaista viivaa käytetään suomen kielessä myös ajatusviivana:
            \begin{lstlisting}
\LaTeX{}ia ei opi ymmärtämään -- siihen
vain tottuu.<>
            \end{lstlisting}
            tulostaa
            \begin{sample}
                \LaTeX{}ia ei opi ymmärtämään -- siihen
                vain tottuu.
            \end{sample}
    \end{itemize}
\end{fframe}

\begin{fframe}
    \begin{harj}
        Lisää dokumenttiisi jokin lainaus käyttäen \cns{quote}- tai \cns{quotation}--ympäristöä. 
    \end{harj}
    \begin{harj}
        Lisää työhösi alaviite komennolla \lstinline-\footnote{}-. 
    \end{harj}
\end{fframe}

\subsubsection{\BibTeX-järjestelmä}
\begin{fframe}
    \frametitle{Kirjallisuusviitteet}
    \LaTeX issa kirjallisuusviittaukset kannattaa hoitaa \BibTeX-järjestelmän avulla. 
    \pause
    \vaihto
%    Tällöin jokaista lähdeteosta kohden luodaan erillinen \verb-.bib--tiedosto, joka sisältää teoksen tiedot. 
    Tällöin lähdeteokset kirjataan erilliseen \cns{.bib}-tiedostoon (tai -tiedostoihin).
    \pause
    \vaihto
    Tiedostot on yksinkertaisinta tallentaa samaan kansioon \cns{.tex}-tiedoston kanssa.
    \pause
    \vaihto
    Viittaaminen tapahtuu komennolla \lstinline-\cite{tunnus}-, jossa \cns{tunnus} on eräs \cns{.bib}-tiedostosta löytyvä merkkijono.
    %\vaihto
    %Aivan työn loppuun luodaan kirjallisuusluettelo komennoilla
    %\begin{Verbatim}[frame=single]
    %\bibliographystyle{luettelon tyyli}
    %\bibliography{teos1.bib, teos2.bib, teos3.bib,...}
    %\end{Verbatim}
    %Edellytyksenä on, että tiedostot teos1.bib jne. ovat samassa kansiossa \verb-.tex--tiedoston kanssa.
\end{fframe}

\begin{fframe}
    \frametitle{.bib-tiedostot}
    \cns{.bib}-tiedoston voi luoda millä tahansa tekstieditorilla, erityisen hyvin Texmakerilla. 
    \pause
    \vaihto
    Tiedoston sisältö voisi olla esimerkiksi seuraava:\vaihto
    \begin{lstlisting}[language=BibTeX,basicstyle=\ttfamily\small]
@book{kemper,
    title={A Course in Commutative Algebra},
    author={Kemper, G.},
    isbn={9783642035456},
    series={Graduate Texts in Mathematics},
    url={http://books.google.fi/books?id=8kxlj48DWM4C},
    year={2010},
    publisher={Springer Berlin Heidelberg}
}<>
    \end{lstlisting}
    %Tiedosto tallennetaan muodossa \verb-jokinNimi.bib- samaan kansioon \verb-.tex--tiedoston kanssa (ei pakollista, mutta yksinkertaisinta).
\end{fframe}

\begin{fframe}
    \frametitle{.bib-tiedostot}
    \begin{harj}
        Kopioi edellisen esimerkin sisältö uuteen tiedostoon ja tallenna se nimellä \cns{lahteet.bib} työkansioosi. 
    \end{harj}
\end{fframe}

\begin{fframe}
    \frametitle{.bib-tiedoston sisältö}
    \begin{itemize}[<+->]
        \item Ensimmäisellä komennolla (\lstinline[language=BibTeX]-@book-, \lstinline[language=BibTeX]-@article-, \lstinline[language=BibTeX]-@unpublished- jne.) kerrotaan, minkä tyyppisestä teoksesta on kyse. 
        \item Aaltosulkeisiin ennen ensimmäistä pilkkua tuleva merkkijono on se tunnus, jolla teokseen viitataan komennolla \lstinline-\cite{tunnus}-. 
        \item Tunnuksen voi valita vapaasti (ilman ääkkösiä ja erikoismerkkejä) eikä se tule näkyviin mihinkään.  Esimerkiksi yllä tunnus on \cns{kemper}, joten teokseen viitattaisiin komennolla \lstinline-\cite{kemper}-.
        \item Muut kentät (ja niiden pakollisuus/valinnaisuus) määräytyvät teoksen tyypin mukaan.
    \end{itemize}
\end{fframe}

\begin{fframe}
    \frametitle{Luokat ja .bib-tiedoston sisältö}
    Kun tiedostoja luodaan käsin, täytyy selvittää mitkä kentät ovat pakollisia ja mitkä valinnaisia. 
    \vaihto
    Esimerkiksi taulukosta 
    \begin{scriptsize}
        \url{http://en.wikibooks.org/wiki/LaTeX/Bibliography_Management#Entry_and_field_types_in_.bib_files}
    \end{scriptsize}
    selviää, että kirjalle (\lstinline[language=BibTeX]-@book-) pakollisia kenttiä ovat \cns{title} ja \cns{author}, muut valinnaisia.
    \pause
    \vaihto
    Lisätietoa käytettävistä teosluokista löytyy esimerkiksi sivulta
    \begin{scriptsize}
        \url{http://en.wikibooks.org/wiki/LaTeX/Bibliography_Management#Standard_templates}.
    \end{scriptsize}
    \vaihto
    Käytännössä jokaista kuviteltavissa olevaa lähdeteosta varten löytyy jokin sopiva luokka (ja tarpeen tullen sellaisen voi luoda itsekin).
\end{fframe}

\begin{fframe}
    \begin{harj}
        Lisää tiedostoon \cns{lahteet.bib} kirja, jonka nimi on Topologia I, kirjoittaja Jussi Väisälä, julkaisija Limes ry ja painovuosi 2000. Valitse viittaustunnukseksi \cns{topo1}.
    \end{harj}
\end{fframe}

\begin{fframe}
    \frametitle{.bib-tiedostot Internetistä}
    Kirjojen teostietoja löytää \BibTeX -muodossa varsin suurella todennäköisyydellä Google Books-palvelusta. 
    \pause
    \vaihto
    Kuhunkin teokseen liittyvän Tietoja teoksesta-sivun alalaidassa on kohta Vie sitaatti ja sen vieressä painike \BibTeX, josta tiedoston voi ladata.
    \pause
    \vaihto
    Tiedosto kannattaa avata Texmakerillä, tehdä mahdolliset muutokset ja tallentaa haluttuun kansioon. 
    \vaihto
    Käytännössä ainakin viittaustunnus on syytä vaihtaa helpommaksi.
\end{fframe}

\begin{fframe}
    \frametitle{.bib-tiedostot Internetistä}
    \begin{harj}
        Etsi Google Books-palvelusta Walter Rudinin kirjan Complex Analysis (kirjan versiolla ei väliä) tiedot \BibTeX -muodossa. Avaa tiedosto Texmakerilla ja valitse viittaustunnus haluamaksesi. Tallenna tiedosto nimellä \cns{Rudin.bib} työkansioosi. (Huomaa, että palvelu tarjoaa tiedostot muodossa \cns{.bibtex}.)
    \end{harj}
\end{fframe}

\begin{fframe}
    \frametitle{Kirjallisuusluettelon luominen}
    Kun tarvittavat \cns{.bib}-tiedostot on tallennettu koneelle, on syytä luoda kirjallisuusluettelo. Kirjallisuusluetteloon tulee näkyviin ne teokset, joiden tiedot ovat olemassa ja joihin on viitattu ainakin kerran. Ei-viitatun teoksen saa lisättyä lähdeluetteloon \lstinline-\nocite{tunnus}--komennolla.
    \pause
    \vaihto
    Luettelo luodaan työn loppuosaan komennoilla
    \begin{lstlisting}
\bibliographystyle{luettelon tyyli}
\bibliography{lista1,lista2,lista3,...}<>
    \end{lstlisting}
    Koodissa \cns{lista\textit{x}} ovat \cns{.bib}-tiedostojen nimet ilman \cns{.bib}-päätettä. Huomaa myös, ettei pilkun jälkeen sallita välilyöntiä.
    \vaihto
    Edellisen koodin toimimisen edellytyksenä on, että tiedostot \cns{lista1.bib} jne. ovat samassa kansiossa \cns{.tex}-tiedoston kanssa. 
\end{fframe}

\begin{fframe}
    \frametitle{Kirjallisuusluettelon luominen}
    \begin{harj}
        Edellä loit \BibTeX-tiedostot \cns{lahteet.bib} ja \cns{Rudin.bib}. Tuo teokset kirjallisuusluetteloon koodilla 
        \begin{lstlisting}
\bibliographystyle{plain}
\bibliography{lahteet,Rudin}<>
        \end{lstlisting}
        Sijoita koodi työsi loppuosaan, esimerkiksi juuri ennen komentoa \lstinline-\end{document}-.
    \end{harj}
    \begin{harj}
        Viittaa kuhunkin teokseen työssäsi komennolla \lstinline-\cite{viittaustunnus}-.
    \end{harj}
\end{fframe}
\begin{fframe}
    \frametitle{Käytäntö}
    %Kun tiedostot on luotu ja tallennettu, niihin voidaan viitata tuloksellisesti jo mainitulla komennolla \verb-\cite{tunnus}-. 
    Jotta viitteet tulisivat näkyviin, toimi seuraavasti:
    \begin{itemize}[<+->]
        \item aja tiedosto PDFLaTeXilla
        \item aja tiedosto \BibTeX{}illa (ja tarkista alapalkkiin ilmestyvät viestit!)
        \item aja tiedosto PDFLaTeXilla kahdesti
    \end{itemize} 
    \pause
    Jos viittaukset eivät tule näkyviin etkä tiedä mistä se johtuu, tarkista seuraavat kohdat:
    \begin{itemize}
        \item oletko käyttänyt viitatessa oikeita tunnuksia?
        \item oletko tuonut tiedostot \LaTeX iin oikein ja oikeilla nimillä?
        \item ovatko tiedostot todella muotoa \cns{.bib}?
        \item ovatko tiedostot oikeassa kansiossa?
    \end{itemize}
\end{fframe}

\begin{fframe}
    \begin{extra}
        \begin{itemize}
            \item Viittauksen lisäämisen/poistamisen/muuttamisen jälkeen ensimmäinen \LaTeX-ajo merkitsee löytyneet viittaustunnukset \cns{.aux}-tiedostoon.
            \item Tämän jälkeen \BibTeX-ajo kirjoittaa \cns{.aux}-tiedoston perusteella lähdeluettelon tuottavan \LaTeX-koodin \cns{.bbl}-tiedostoon.
            \item Seuraava \LaTeX-ajo lukee myös \cns{.bbl}-tiedoston koodin, jolloin lähdeluettelo tulee näkyviin (jos sitä ei vielä ole) ja viittaustiedot lisätään/päivitetään \cns{.aux}-tiedostoon.
            \item Viimeinen \LaTeX-ajo osaa nyt \cns{.aux}-tiedoston perusteella korvata kysymysmerkit oikeilla viittauksilla.
        \end{itemize}
    \end{extra}
\end{fframe}
%\begin{fframe}
%\frametitle{Viittaustyyli}
%%\begin{harj}
%Luo tai etsi Internetistä kolmelle eri teokselle \BibTeX-tiedostot ja viittaa teoksiin onnistuuneesti. Valitse teoksille eri luokat, esim \verb-@book-, \verb-@misc- ja \verb-@unpublished-.
%\end{harj}

%\end{fframe}
\subsubsection{Viittaustyyli}
\begin{fframe}
    \frametitle{Viittaustyyli}
    Viittaukset tulevat näkyviin hakasuluissa olevina numeroina. Esimerkiksi lähdeteoksen kirjoittajaa varten lukija joutuu siis tarkistamaan lähdeluettelon. Tämä on tyypillistä varsinkin matemaattisessa kirjallisuudessa. 
    \vaihto
    Toisenlaista viittaustekniikkaa varten voi käyttää esimerkiksi \cns{natbib}-järjestelmää, jota ei kuitenkaan tällä kurssillä käsitellä. Lisätietoa saa vaikkapa osoitteesta
    \begin{scriptsize}
        \url{http://en.wikibooks.org/wiki/LaTeX/Bibliography_Management#Natbib}
    \end{scriptsize}
\end{fframe}


\subsection{Dokumentin viimeistely}

\subsubsection{Sisällysluettelo ja kansilehti}
\begin{fframe}
    \frametitle{Sisällysluettelo ja kansilehti}
    Sisällysluettelon saa haluamaansa kohtaan työtä komennolla \lstinline-\tableofcontents-. Tiedoston joutuu yleensä ajamaan useaan kertaan, ennen kuin numeroinnit tulevat näkyviin oikein.
    \pause
    \vaihto
    Kansilehti luodaan komennolla \lstinline-\maketitle-. Tätä varten esittelyosaan lisätään komennot \lstinline-\title{otsikko}-, \lstinline-\author{nimi}- ja haluttaessa \lstinline-\date{pvm}-. Päivämäärän saa pois komennolla \lstinline-\date{}-.
    \vaihto
    Kansilehti tulostuu joissain luokissa (esim. \cns{book} ja \cns{report}) omaksi sivukseen, \cns{article}-luokassa osaksi ensimmäistä sivua.
\end{fframe}

\begin{fframe}
    \begin{harj}
        Luo työllesi kansilehti ja sisällysluettelo. Kokeile miltä työsi näyttää, jos vaihdat luokaksi \cns{report} tai \cns{book}. Millaisia eroja huomaat? 
    \end{harj}
    \begin{harj}
        \begin{itemize}
            \item Dokumenttia on helpompi navigoida elektronisesti, kun viitteet toimivat linkkeinä. Ota tätä varten käyttöön paketti \cns{hyperref}. Paketti kannattaa ladata kaikkien muiden pakettien jälkeen.
            \item Kokeile kääntää dokumentti -- linkkien ympärillä on nyt ruma laatikko. Paranna tilannetta antamalla \cns{hyperref}-paketille argumentti \cns{colorlinks=true} (ota mallia esim. \cns{inputenc}-paketin lataavasta rivistä).
\end{itemize}
    \end{harj}
\end{fframe}

\subsubsection{Dokumenttiluokan asetukset}
\begin{fframe}
    \frametitle{Dokumenttiluokan asetukset}
    Tiedoston aloittavalle komennolle \lstinline-\documentclass[]{}- voi antaa valinnaisena argumenttina eräitä dokumentin ulkoasua koskevia asetuksia:
    \begin{itemize}[<+->]
        \item Kirjainkoon vaihtoehdot ovat \cns{10pt}, \cns{11pt} ja \cns{12pt}. 
        \item Luonnostilan saa valinnalla \cns{draft}. Tällöin kääntäminen on nopeaa ja liian pitkät rivit tulevat merkityiksi.
        \item Tekstin saa jaettua kahteen kolumniin valinnalla \cns{twocolumn}. 
        \item article-luokassa tulostus on oletusarvoisesti yksipuoleinen -- tämän voi vaihtaa valinnalla \cns{twoside} (vast. \cns{oneside}).
    \end{itemize}
    \pause
    Useammat määritteet erotetaan toisistaan pilkuilla, kuten tavallista.
    \vaihto
    Hieman lisätietoa löytyy esimerkiksi osoitteesta \begin{scriptsize}
        \url{http://texblog.org/2013/02/13/latex-documentclass-options-illustrated}.
    \end{scriptsize}
\end{fframe}

