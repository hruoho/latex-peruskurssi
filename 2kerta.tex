\section{2. kerta}


\subsection{Matemaattinen teksti}
\begin{fframe}
    \frametitle{Sulut}
    Varsinkin kaavariville kirjoitettaessa monet symbolit ovat huomattavan kookkaita. Sulut tulostuvat automaattisesti oikean kokoisina kun niille käytetään komentoja pelkkien merkkien \lstinline-(- ja \lstinline-)- sijaan. \vaihto
    \begin{tabular}{cc|cc}
        %\multicolumn{2}{c}{Vasen}&\multicolumn{2}{c}{Oikea}\\
        Komento & Tulostus & Komento & Tulostus\\
        \hline
        \lstinline-\left(- & \(\left(\right.\) & \lstinline-\right)-& \(\left.\right)\)\\
        \lstinline-\left[- & \(\left[\right.\) & \lstinline-\right]-& \(\left.\right]\)\\
        \lstinline-\left\lbrace- & \(\left\lbrace\right.\) & \lstinline-\right\rbrace- & \(\left.\right\rbrace\)\\
        \lstinline-\left\langle- & \(\left\langle\right.\) & \lstinline-\right\rangle- & \(\left.\right\rangle\)\\
        \lstinline-\left|- & \(\left|\right.\) & \lstinline-\right|- & \(\left.\right|\)\\
    \end{tabular}
    \vaihto
    Jokaista \lstinline-\left--alkuista komentoa täytyy seurata \lstinline-\right--alkuinen komento, vähintään \lstinline-\right.- , joka ei tulosta mitään. Samoin \lstinline-\right--alkuista komentoa on edellettävä \lstinline-\left--alkuinen komento, vähintään \lstinline-\left.- . 
\end{fframe}

\begin{fframe}
    \begin{harj}
        Tuota seuraava dokumenttiisi:
        \begin{sample}
            Todista seuraava yhtälö:
            \[
                \left\lbrace 2^n \,\middle|\, n\in\Z\right\rbrace=\left\lbrace \left(\frac{1}{2}\right)^{-n}\,\middle|\, n\in\Z\right\rbrace
            \]
        \end{sample}
        Komennolla \lstinline-\middle|- saat pystyviivan automaattisesti oikean kokoisena ja komennolla \lstinline-\,- hieman ylimääräistä väliä niiden ympärille. Tehtävästä \ref{rajaArvo} saat apua symboleiden \(\in\) ja \(\Z\) luomiseen.
    \end{harj}
\end{fframe}
\begin{fframe}
    \begin{harj}
        Kirjoita seuraava:
        \begin{sample}
            Jos \(F\) on \(\sigma\)-algebra ja \(P\colon F\to\R\) todennäköisyys, niin tapahtumille \(A_1,A_2,\dotsc\in F\) pätee
            \[
                P\left(\bigcup_{i=1}^\infty A_i\right) \leq \sum_{i=1}^\infty P(A_i).
            \]
        \end{sample}
    \end{harj}
\end{fframe}

\begin{fframe}
    \frametitle{Vanhoja tapoja}

    \begin{itemize}[<+->]
        \item \emph{Vanha \TeX-tapa} kirjoittaa kaava tekstin sekaan on \lstinline-$kaava$-:
            \begin{lstlisting}
Yhtälön $2x+1=0$ ratkaisu on $x=-1/2$.<>
            \end{lstlisting}
        \item Käännöksissä tuleva virheilmoitus ''\texttt{Missing \$ inserted.}'' tulee tästä -- se tarkoittaa, että yritit kirjoittaa joko 
            \begin{itemize}
                \item matematiikkaa kuten \lstinline-\frac{}{}- olematta matematiikkatilassa tai
                \item ei-matematiikkaa kuten \lstinline-\section{}- matematiikkatilassa.
            \end{itemize}
        \item \TeX{} yritti itse väliaikaisesti korjata virheen lisäämällä dollarin jonnekin.
        \item Käännös saattaa tämän paikkauksen takia mennä läpi, mutta kyseessä on silti virhe, joka on korjattava!
        \item \TeX-tapa kaavarivin tuottamiseen on \verb-$$kaava$$-. \alert{Sitä ei pitäisi \LaTeX{}issa käyttää ollenkaan}, vaikka jotkut niin tekevätkin\dots
    \end{itemize}
\end{fframe}

\begin{fframe}
    \frametitle{Pitkät kaavat}
    Toisinaan matemaattiset ilmaisut ovat niin pitkiä, etteivät ne mahdu yhdelle riville. Tällöin voidaan käyttää esimerkiksi ympäristöä \cns{align*}, jolla rivit saadaan allekkain seuraavasti:

    \begin{minipage}{4cm}
        \begin{lstlisting} 
\begin{align*}
    xyz &= abc\\
        &= bca\\
      w &= cab
\end{align*}<>
        \end{lstlisting}
    \end{minipage}
    \begin{minipage}{4cm}
        \begin{serif}
            \begin{align*}
                xyz &= abc\\
                    &= bca\\
                  w &= cab
            \end{align*}
        \end{serif}
    \end{minipage}

    Huomaa, ettei \cns{align*}-ympäristöä tarvitse erikseen sijoittaa matematiikkatilaan.
\end{fframe}

\begin{fframe}
    \frametitle{Pitkät kaavat}
    Ympäristölle \cns{align*} rivin katkaisukohta kerrotaan komennolla \lstinline-\\- ja tasauskohta merkillä \lstinline-&-. Katkaisukohta täytyy löytyä kaikilta paitsi viimeiseltä riviltä. Sen sijaan tasauskohdan täytyy löytyä jokaiselta riviltä!
    \vaihto
    Pitkille kaavoille voi vaihtoehtoisesti käyttää ympäristöä \cns{multline*}, jolle kerrotaan vain rivien katkaisukohdat. Se asettelee rivit automaattisesti (yleensä vähän epämääräisesti).
    \vaihto
    Myös tähdettömiä ympäristöjä \cns{align} ja \cns{multline} voi käyttää, jolloin jokainen rivi tulee numeroiduksi.
    %\begin{minipage}{3cm}
    %\begin{scriptsize}
    %\begin{Verbatim}[frame=single]
    %\begin{align*}
    %ax^2+bx+c &= 0\\
    %4a^2x^2+4abx+4ac &= 0\\
    %4a^2x^2+4abx &= -4ac\\
    %4a^2x^2+4abx +b^2 &= b^2-4ac\\
    %(2ax+b)^2 &= b^2 -4ac\\
    %2ax+b &= \pm\sqrt{b^2-4ac}\\
    %x &= \frac{-b\pm\sqrt{b^2-4ac}}{2a}
    %\end{align*}
    %\end{Verbatim}
    %\end{scriptsize}
    %\end{minipage}
    %\begin{minipage}{3cm}
    %\begin{align*}
    %ax^2+bx+c &= 0\\
    %4a^2x^2+4abx+4ac &= 0\\
    %4a^2x^2+4abx &= -4ac\\
    %4a^2x^2+4abx+b^2 &= b^2-4ac\\
    %(2ax+b)^2 &= b^2 -4ac\\
    %2ax+b &= \pm\sqrt{b^2-4ac}\\
    %x &= \frac{-b\pm\sqrt{b^2-4ac}}{2a}
    %\end{align*}
    %\end{minipage}
\end{fframe}

%\begin{fframe}
%Esimerkiksi komennolla
%\begin{scriptsize}
%\begin{Verbatim}[frame=single]
%\begin{align*}
%(3\vv-\vw)\cdot(\vv+\vw)
%&=(3\vv-\vw)\cdot\vv+ (3\vv-\vw)\cdot\vw \\
%&=3\vv\cdot\vv-\vw\cdot\vv+ 3\vv\cdot\vw - \vw\cdot\vw\\
%&=3||\vv||^2+2(\vv\cdot\vw)-||w||^2\\
%&=3\cdot 2^2+2\cdot (-1)-3^2=1.
%\end{align*}
%\end{Verbatim}
%\end{scriptsize}
%saa seuraavan yhtälöketjun kirjoitetuksi allekkain, yhtäsuuruusmerkit kohdakkain:
%\begin{align*}
%(3\vv-\vw)\cdot(\vv+\vw)&=(3\vv-\vw)\cdot\vv+ (3\vv-\vw)\cdot\vw \\
%&=3\vv\cdot\vv-\vw\cdot\vv+ 3\vv\cdot\vw - \vw\cdot\vw\\
%&=3||\vv||^2+2(\vv\cdot\vw)-||w||^2\\
%&=3\cdot 2^2+2\cdot (-1)-3^2=1.
%\end{align*}
%
%\end{fframe}

\begin{fframe}
    %Seuraavassa pohja \verb-align*--ympäristön käyttämistä varten:
    %\vaihto
    %\begin{minipage}{4cm}
    %\begin{scriptsize}
    %\begin{Verbatim}[frame=single]
    %\begin{align*}
    %xyz &= abc\\
    %	&= bca\\
    %	&= cab\\
    %\end{align*}
    %\end{Verbatim}
    %\end{scriptsize}
    %\end{minipage}
    \begin{harj}
        Kirjoita muutaman yhtälön ketju ja sijoita yhtälöt allekkain, tasaten haluamastasi kohdasta. Voit esimerkiksi derivoida vaiheittain funktion \(x^3\sin(\cos(x))\) tai keksiä jotkin muut yhtälöt. Yhtälöiden ei tarvitse olla tosia. Komennolla \lstinline-\sin- saa sinifunktion tulostettua pystyfontilla.
    \end{harj}
    \begin{harj}
        Kopioi osa edellisen tehtävän yhtälöketjua ja kirjoita se numeroituun tai numeroimattomaan \cns{multline}-ympäristöön. 
    \end{harj}
\end{fframe}

\begin{fframe}
    \frametitle{Numeroidut kaavat}
Komennolla \lstinline-\begin{equation}...\end{equation}- saa luotua samanlaisen kaavarivin kuin komennolla \lstinline-\[...\]-, mutta numeroinnilla varustettuna. \pause Esimerkiksi

    \begin{lstlisting}
Einsteinin yhtälöön
\begin{equation}
    E=mc^2
\end{equation}
viitataan myöhemmin.<>
    \end{lstlisting}
    tuottaa seuraavan:
    \begin{sample}
        Einsteinin yhtälöön
        \begin{equation*}
            \hskip\textwidth minus \textwidth E=mc^2 \hskip\textwidth minus \textwidth (1)
        \end{equation*}
        viitataan myöhemmin.
    \end{sample}
    \pause
    Numeroituihin kaavoihin viittaamiseen palataan tuonnempana.
\end{fframe}

\begin{fframe}
    \begin{harj}
        \label{viittausTehtava}
        Tuota seuraava dokumenttiisi:
        \begin{sample}
            Eksponenttifunktiolla \(e^x\) on sarjakehitelmä
            \begin{equation*}
                \hskip\textwidth minus \textwidth e^x=\sum_{n=1}^\infty\frac{x^n}{n!}.\hskip\textwidth minus \textwidth (1)
            \end{equation*}
        \end{sample}
        Summamerkinnän saat komennolla \lstinline-\sum_{}^{}-, osamäärän komennolla \lstinline-\frac{}{}- ja äärettömän symbolin komennolla \lstinline-\infty-. 
    \end{harj}
\end{fframe}


\subsection{Lauseympäristö}
\begin{fframe}
    \frametitle{Lauseympäristö}
    Paketti \cns{amsthm} tarjoaa mahdollisuuden esittää mm. lauseet, lemmat ja määritelmät tyylikkäästi. 
    \vaihto
    Tarvittavat ympäristöt määritellään esittelyosassa komennolla \lstinline-\newtheorem{}{}-. Ensimmäisiin aaltosulkeisiin tulee nimi, jolla ympäristöä käytetään ja jälkimmäisiin ympäristön otsikko, joka halutaan näkyväksi lopullisessa työssä.
    \vaihto
Komento \lstinline-\newtheorem{esim}{Esimerkki}- loisi ympäristön, jota käytettäisiin komennolla \lstinline-\begin{esim}...\end{esim}- ja jonka otsikko valmiissa työssä olisi Esimerkki. 
    %\begin{Verbatim}[frame=single]
    %\newtheorem{esim}{Esimerkki}
    %\newtheorem{lause}{Lause}
    %\newtheorem{maar}{Määritelmä}
    %\end{Verbatim}
\end{fframe}

\begin{fframe}
    \begin{harj}
        \label{harjYmparistot}
        Luo esittelyosassa ainakin ympäristöt Lause, Määritelmä ja Esimerkki. Käytä luomiasi ympäristöjä ainakin kerran. 
    \end{harj}
    \begin{harj}
        Lauseiden todistuksia varten on oma ympäristönsä, jota käytetään komennoilla \lstinline-\begin{proof}...\end{proof}-.
        Kirjoita edellisessä harjoituksessa luomallesi lauseelle jokin todistus. Todistukseksi kelpaa muutama rivi valitsemaasi tekstiä.
    \end{harj}
    %Esimerkiksi rivi
    %\begin{Verbatim}[frame=single]
    %\begin{esim}
    %Tämä on esimerkki\ldots
    %\end{esim}
    %\end{Verbatim}
    %\pause
    %tuottaisi nyt ympäristön
    %\begin{framed}
    %\begin{esim}
    %Tämä on esimerkki\ldots
    %\end{esim}
    %\end{framed}
    %Jos ympäristöä \verb-esim- ei ole määritelty esittelyosassa, tämä aiheuttaa virheen!
\end{fframe}

%\begin{fframe}
%\pause
%Vastaavasti esimerkiksi komento
%\begin{Verbatim}[frame=single]
%\begin{lause}
%Tämä on lause\ldots
%\end{lause}
%\end{Verbatim}
%tuottaa ympäristön
%\pause
%\begin{framed}
%\begin{lause}
%Tämä on lause\ldots
%\end{lause}
%\end{framed}
%\pause
%Kokeile, miltä nämä näyttävät omassa työssäsi.
%\end{fframe}

\begin{fframe}
    \frametitle{Lauseympäristön tyyli}
    Lauseympäristön tyyli valitaan esittelyosassa komennolla \lstinline-\theoremstyle{...}-. Valittavissa on tyylit \cns{plain}, \cns{definition} ja \cns{remark}. Tyylin valinta vaikuttaa sitä seuraaviin komennolla \lstinline-\newtheorem{}{}- luotuihin ympäristöihin.
    \vaihto
    Esimerkiksi kirjoittamalla
    \pause
    \begin{lstlisting}
\theoremstyle{plain}
\newtheorem{lemma}{Lemma}
\newtheorem{lause}{Lause}

\theoremstyle{definition}
\newtheorem{maar}{Määritelmä}<>
    \end{lstlisting}
    \pause
    ympäristöt \cns{lemma} ja \cns{lause} tulostuvat \cns{plain}-tyylin mukaisesti ja ympäristö \cns{maar} \cns{definition}-tyylin mukaisesti.
\end{fframe}

\begin{fframe}
    \begin{harj}
        Valitse harjoituksessa \ref{harjYmparistot} luomillesi lauseympäristöille jotkin tyylit. Kokeile, miltä erilaiset tyylit näyttävät ja valitse mieleisesi!
    \end{harj}
\end{fframe}

\begin{fframe}
    \frametitle{Lauseympäristöjen numerointi}
    Komennolla \lstinline-\newtheorem{}{}- luotu ympäristö luo samalla uuden \newterm{laskurin}, jonka mukaan ympäristön toteutumat numeroidaan. 
    \vaihto
    Laskuri voidaan myös asettaa toiselle laskurille \newterm{alisteiseksi} tai valita mielivaltaisesti. Erityisesti eri ympäristöt voivat käyttää samaa laskuria, jos niin halutaan.
    \vaihto
    Komennolla \lstinline-\newtheorem{}{}[]- on valinnaisena argumenttina laskuri, jolle ympäristön numerointi halutaan alisteiseksi. Tämä voisi olla esim. laskuri \cns{section}, jolloin ympäristö numeroitaisiin muodossa ''osionNumero.ympäristönNumero''.
\end{fframe}

\begin{fframe}
    \frametitle{Lauseympäristöjen numerointi}
    Jos ympäristön halutaan käyttävän jotain tiettyä laskuria, käytetään komentoa \lstinline-\newtheorem{}[laskurin nimi]{}-. Huomaa, että tässä valinnainen argumentti sijoittuu pakollisten väliin. 
    \vaihto
    Esimerkiksi koodilla
    \begin{lstlisting}
\newtheorem{teor}{Teoreema}[section]
\newtheorem{lemma}[teor]{Lemma}<>
    \end{lstlisting}
    luodut Lemma- ja Teoreema-ympäristöt noudattavat samaa, laskurille \cns{section} alisteista numerointia.
    %\begin{framed}
    %\setcounter{laskuri}{0}
    %\begin{trma}
    %\ldots
    %\end{trma}
    %\begin{prop}
    %\ldots
    %\end{prop}
    %\end{framed}
\end{fframe}

\begin{fframe}
    Kokonaan numeroimattoman lauseympäristön voi luoda komennolla \lstinline-\newtheorem*{}{}-.
    \begin{harj}
        Aseta yksi lauseympäristöistäsi (esim. Lause) laskurille \cns{section} alisteiseksi.
        %Anna yhdelle lauseympäristöistäisi numeroinniksi \verb-\section- tai \verb-\subsection-.
        Anna sitten toisen lauseympäristön (esim. Lemma) laskuriksi edellinen lauseympäristö. 
    \end{harj}
    \begin{harj}
        Luo jokin numeroimaton lauseympäristö ja käytä sitä työssäsi.
    \end{harj}
\end{fframe}


\subsection{Sisäiset viittaukset}
\begin{fframe}
    \frametitle{Sisäiset viittaukset}
    \LaTeX illa voi helposti viitata numeroituihin kohteisiin, kuten lauseisiin tai yhtälöihin. Viitattava kohde täytyy ensin nimetä komennolla \lstinline-\label{nimi}- (nimi ei tulostu työhön). Tämän jälkeen viittaaminen onnistuu komennoilla 
    \begin{itemize}
        \item \lstinline-\ref{nimi}- (tulostaa viitattavan kohteen numeron)
        \item \lstinline-\eqref{nimi}- (tulostaa kohteen numeron sulkujen sisällä)
        \item \lstinline-\pageref{nimi}- (tulostaa sen sivun numeron, jolla kohde on)  
    \end{itemize}
    Sisäisten viittausten kanssa tulee aina käyttää komentoja. Tällöin viittaukset pysyvät kohdallaan vaikka numeroinnit muuttuisivat työn edetessä.
\end{fframe}

\begin{fframe}
    \frametitle{Esimerkki}
    Seuraavassa on osiolle ''Asiaa'' annettu tunnus \cns{sec:asiaa}. Osioon viitataan muualla tekstissä komennolla \lstinline-\ref-.
    \begin{lstlisting}
\section{Asiaa}\label{sec:asiaa}
Jotain asiaa\dots

% Muualla koodissa
Osiossa~\ref{sec:asiaa} puhuttiin asiaa.<>
    \end{lstlisting}
    \pause
    Numeroidulle yhtälölle tai vaikkapa lauseelle tunnus annetaan ympäristön aloittavan komennon jälkeen:
    \begin{lstlisting}
\begin{equation}\label{eq:tunnus}
    ...
\end{equation}<>
    \end{lstlisting}
\end{fframe}

\begin{fframe}
    Huom! Uuden viittauksen jälkeen työn joutuu ajamaan kahdesti, jotta numeroinnit tulevat näkyviin (kahden kysymysmerkin sijaan).
    \begin{extra}
        Jos \LaTeX-tiedoston nimi on \cns{nimi.tex}, niin ensimmäisellä ajokerralla \LaTeX\ luo (tai päivittää) tiedoston \cns{nimi.aux}, joka sisältää viittaustiedot (kokeile avata tiedosto). Toisella ajokerralla viittaustiedot luetaan tästä ja päivitetään lopulliseen tiedostoon. Tätä ei voi yhdellä ajokerralla tehdä, koska viittaukset voivat viitata myös tekstissä eteenpäin.
    \end{extra}
    \begin{harj}
        Viittaa harjoituksessa \ref{viittausTehtava} kirjoittamaasi numeroituun yhtälöön. Käytä komentoja \lstinline-\eqref{}- ja \lstinline-\pageref{}-.
        %\vaihto
        %Huomaa, että viitattavalle kohteelle on ensin annettava tunnus komennolla \lstinline-\label{valitsemasi tunnus}-. Tämä kirjoitetaan ympäristön aloittavan komennon \lstinline-\begin{}- jälkeen. 
    \end{harj}
\end{fframe}


\subsection{Listarakenteet}
\begin{fframe}
    \frametitle{Listarakenteet}
    \LaTeX illa voi luoda listoja mm. ympäristöjen \cns{itemize} ja \cns{enumerate} avulla. Ensimmäinen on ranskalaiset viivat-tyyppinen, jälkimmäinen numeroi listan jäsenet. Näitä käytetään seuraavasti:
    \vaihto
    \begin{minipage}{5cm}
        \begin{lstlisting}
Tässä listani:
\begin{itemize}
    \item asia
    \item toinen asia
    \item kolmas asia
\end{itemize}<>
        \end{lstlisting}
    \end{minipage}
    \begin{minipage}{5cm}
        \begin{serif}
            Tässä listani:
            \begin{itemize}
                \item[\textcolor{black}{\textbullet}] asia
                \item[\textcolor{black}{\textbullet}] toinen asia
                \item[\textcolor{black}{\textbullet}] kolmas asia
            \end{itemize}
        \end{serif}
    \end{minipage}
    \vaihto
    Luettelointiin käytetyt symbolit voi tarvittaessa valita vapaasti. 
\end{fframe}

\begin{fframe}
    \frametitle{Listat}
    Seuraavassa tehtävässä harjoitellaan sisäkkäisten listojen käyttöä.
    \begin{harj}
        Luo ainakin kolmen kohdan numeroitu lista haluamistasi asioista. Luo yhdeksi listan jäseneksi toinen lista ja yhdeksi tämän listan jäseneksi kolmas lista. Käytä (ainakin) viimeiseen listaan numeroimatonta \cns{itemize}-ympäristöä.
    \end{harj}
    \begin{harj}
    Luo vielä yksi lista, mutta käytä tällä kertaa ympäristöä \cns{description}. Se toimii kuten \cns{itemize}, mutta pelkän komennon \lstinline-\item- sijaan käytetään komentoa \lstinline-\item[nimi]-, jossa \cns{nimi} on vapaasti valitsemasi merkkijono.
        %\begin{scriptsize}
        %\begin{Verbatim}[frame=single]
        %\begin{description}
        %  \item[T1] The first item
        %  \item[T2] The second item
        %  \item[T3] The third etc \ldots
        %\end{description}
        %\end{scriptsize}
    \end{harj}
    %Listaukseen käytettäviin symboleihin voi itse vaikuttaa. Opetellaan myöhemmin laskareissa usein käytettävä aakkostettu lista.
\end{fframe}


\subsection{Omat komennot}
\begin{fframe}
    \frametitle{Omat komennot}
    %\LaTeX in käyttämistä helpottaa se, että kaikki on koodattavissa komennoiksi. 
    %Esimerkiksi olisi raskasta kirjoittaa jokainen ristitulon merkki pikseli pikseliltä, joten tälle kannattaa luoda oma komentonsa, jolla asia hoituu automaattisesti.
    Omia komentoja luodaan esittelyosassa komennolla \lstinline-\newcommand{}{}-. Ensimmäinen argumentti on komennon nimi, toiseksi argumentiksi tulee komennon sisältö. Esimerkiksi
    \begin{lstlisting}
\newcommand{\R}{\mathbb{R}}<>
    \end{lstlisting}
    luo komennon \lstinline-\R-, joka tulostaa matematiikkatilassa symbolin \(\R\). Komennon luomisen jälkeen kyseisen symbolin tuottaminen onnistuu helposti.
    \vaihto
    Omia komentoja luomalla voit yksinkertaistaa ja helpottaa omaa työtäsi. Jo parikin kertaa toistuva komentojen ketju kannattaa määritellä esittelyosassa yhdeksi yksinkertaiseksi komennoksi.
\end{fframe}

\begin{fframe}
    \begin{harj}
        Luo komennot merkinnöille \(\N\), \(\Z\), \(\Q\), \(\R\) ja \(\C\). Kirjoita seuraava:
        \begin{sample}
            Lukujoukot muodostavat tornin
            \[
                \N\subset \Z\subset \Q\subset \R\subset \C.
            \]
        \end{sample}
        Osajoukkorelaation saat komennolla \lstinline-\subset-. Kokeile myös, mitä komennot \lstinline-\subseteq-, \lstinline-\subsetneq- ja  \lstinline-\supset- tuottavat. Miten saisit symbolit \(\supsetneq\) ja  \(\supseteq\)? Entä symbolin \(\not\subset\) (tähän tarvitset kaksi komentoa)?
        \vaihto
        Kirjoita työhösi vielä seuraava: \(\Z\not\subset\N\).
    \end{harj}
\end{fframe}

\begin{fframe}
    \frametitle{Omat komennot}
    Komennolla \lstinline-\newcommand{}[]{}- on valinnaisena argumenttina luotavan komennon argumenttien lukumäärä. Esimerkiksi
    \begin{lstlisting}
\newcommand{\set}[1]{ \left\lbrace #1 \right\rbrace }<>
    \end{lstlisting}
    loisi komennon \lstinline-\set{}-, jolla voisi tuottaa joukkomerkinnän. 
    \vaihto
    \begin{minipage}{5.3cm}
        \begin{lstlisting}
Yleensä
\(
 \N = \set{0,1,2,3,\dots}
\)<>
        \end{lstlisting}
    \end{minipage}
    \begin{minipage}{4.7cm}
        \begin{serif}
            Yleensä
            \(
                \N = \set{0,1,2,3,\dots}
            \)
        \end{serif}
    \end{minipage}
    \pause
    \vaihto
    Komennon \lstinline-\newcommand{}{}- käyttäminen aiheuttaa konfliktin, jos samanniminen komento on jo käytössä. Tällöin kannattaa nimetä oma komentonsa toisin. 
    %
%\verb-\newcommand{\komento}[2]{...}- loisi komennon, jota käytettäisiin muodossa \verb-\komento{}{}-. 
    %
    %
    %
    % Esimerkiksi 
    %\begin{scriptsize}
    %\begin{Verbatim}[frame=single]
    %\newcommand{\bino}[3]{\binom{#1}{#3}{#2}^{#3}\left(1-#2\right)^{#1-#3}}
    %\end{Verbatim}
    %
    %\end{scriptsize}
%luo komennon \verb-\bino-, jolla on kolme argumenttia: komento \verb-\bino{n}{p}{k}- tuottaa binomitodennäköisyyden kaavan \[\bino{n}{p}{k}.\]
\end{fframe}

%\begin{fframe}
%    \frametitle{Omat komennot}
%    Komennon \lstinline-\newcommand{}{}- käyttäminen aiheuttaa konfliktin, jos samanniminen komento on jo käytössä. Tällöin kannattaa nimetä oma komentonsa toisin. 
%    \vaihto
%    \begin{extra}
%        Jos välttämättä haluaa korvata valmiin komennon omallaan, voi käyttää komentoa \verb-\renewcommand{}{}-. Tämä on joskus tarpeen, mutta saattaa sotkea pahasti asioita!
%    \end{extra}
%\end{fframe}

\begin{fframe}
    \begin{harj}
        \LaTeX issa ei ole valmista komentoa itseisarvofunktiota varten. Korjaa puute luomalla komento \lstinline-\abs{}-, jonka argumentti on itseisarvomerkkien sisään tuleva lauseke. Tuota sen avulla seuraavat kolmioepäyhtälöt:
        \begin{sample}
            \[
                \abs{\abs{x}-\abs{y}}\leq \abs{x+y}\leq\abs{x}+\abs{y}
            \]
            ja
            \[
                \abs{\int_a^b f(x)\,dx} \leq\int_a^b\abs{f(x)}\, dx.
            \]
        \end{sample}
        Huomaa, että itseisarvomerkkien koon on syytä muuttua niiden sisältämän lausekkeen koon mukaan.\vaihto Määrätyn integraalin saat komennolla \lstinline-\int_{}^{}- ja pienen välin komennolla \lstinline-\,- (ennen merkkiä \(dx\)).
    \end{harj}
\end{fframe}

