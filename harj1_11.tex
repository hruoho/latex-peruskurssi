    \begin{harj}
        Tuota seuraava dokumenttiisi:
        \begin{sample}
            Jos \(F\) on \(\sigma\)-algebra ja \(A_i\in F\) kaikilla \(i= 1,2,\dots\), niin 
            \[
                \bigcup_{i=1}^{\infty} A_i\in F.
            \]
        \end{sample}
        Tarvitset mm. komentoja \verb-\sigma-, \verb-\in-, \verb-\bigcup- ja \verb-\infty-. Ala- ja yläindeksit kirjoitetaan merkkien \verb-_- ja \verb-^- avulla (jos indeksiin halutaan enemmän kuin yksi merkki, se täytyy laittaa aaltosulkeisiin kuten edellisessä tehtävässä). 
    \end{harj}
