    \begin{harj}
        Kirjoita seuraava:
        \begin{sample}
            Jos \(F\) on \(\sigma\)-algebra ja \(P\colon F\to\R\) todennäköisyys, niin tapahtumille \(A_1,A_2,\dotsc\in F\) pätee
            \[
                P\left(\bigcup_{i=1}^\infty A_i\right) \leq \sum_{i=1}^\infty P(A_i).
            \]
        \end{sample}
        \begin{extra}
            \begin{scriptsize}
                Tulet todennäköisesti joskus törmäämään myös seuraaviin tapoihin kirjoittaa kaavoja: \verb-$kaava$- ja \verb-$$kaava$$-. Ensimmäinen on käytännössä sama kuin \verb-\(kaava\)-, mutta jos toinen \verb-$- unohtuu, niin syntyvät virheilmoitukset voivat olla vaikeaselkoisia. Toinen taas on \TeX -tapa kirjoittaa kaavarivejä. Se toimii usein samalla tavalla kuin \verb-\[kaava\]-, mutta ei aina, eikä sitä pitäisi \LaTeX issa käyttää.
            \end{scriptsize}
        \end{extra}
    \end{harj}
