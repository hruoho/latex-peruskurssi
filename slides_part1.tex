\documentclass[handout,%
hyperref={unicode,colorlinks=true}]{beamer}
\usecolortheme{seahorse}
\usepackage{etex}   % lisää rekistereitä
%\usefonttheme{serif}
\usefonttheme[onlymath]{serif}     % serif vain matikkamoodiin
\usepackage[utf8]{inputenc}
\usepackage[T1]{fontenc}
\usepackage[finnish]{babel}
\usepackage{amsmath}
%\usepackage{amsfonts}
\usepackage{amssymb}
\usepackage{tikz}
\usepackage{framed}
\usepackage{alltt}
\usepackage{fancyvrb}
\usepackage{multirow}
\usepackage{mathtools}
\usepackage{multicol}
\usetikzlibrary{matrix,arrows,decorations.pathmorphing}
\usetikzlibrary{fadings,patterns,arrows}
\pgfmathsetmacro{\inf}{-2}
\pgfmathsetmacro{\sup}{2}
\tikzstyle{MyPlotStyle}=[domain=\inf:\sup,samples=100,smooth]
\usepgflibrary{arrows}

\usepackage{float}

\usepackage{lmodern}
\usepackage{microtype}
\usepackage{listings}
\lstset{
    language=[LaTeX]TeX,
    basicstyle=\ttfamily,%\footnotesize,
    commentstyle=\ttfamily,%\footnotesize,
    columns=fullflexible,
    keepspaces=true,
    backgroundcolor=\color{gray!20},
    escapechar=<>,
    morekeywords={\part,\chapter,\subsection,\subsubsection,\paragraph,\subparagraph,%
                  \middle,\theoremstyle,\eqref,\subsetneq},
    literate={ä}{{\"a}}1
             {Ä}{{\"A}}1
             {ö}{{\"o}}1
             {Ö}{{\"O}}1
}
\lstdefinelanguage{BibTeX}
    {keywords={%
        @article,@book,@collectedbook,@conference,@electronic,@ieeetranbstctl,%
        @inbook,@incollectedbook,@incollection,@injournal,@inproceedings,%
        @manual,@mastersthesis,@misc,@patent,@periodical,@phdthesis,@preamble,%
        @proceedings,@standard,@string,@techreport,@unpublished%
    },
    comment=[l][\itshape]{@comment},
    sensitive=false,
}

\newcommand{\cns}[1]{\texttt{#1}}   % vakioille yms.
\newcommand{\newterm}[1]{\textit{#1}}

\newcounter{harkka}
\newtheorem{lause}{Lause}
\newtheorem*{huom}{Huomautus}
\newcounter{laskuri}
\newtheorem{trma}{Teoreema}[laskuri]
\newtheorem{prop}[trma]{Propositio}
\theoremstyle{remark}
\newtheorem{harj}{Harjoitus}[section]

\newenvironment{fframe}
    {\begin{frame}[fragile,environment=fframe]}
    {\end{frame}}

\newtheorem*{esim}{Esimerkki}
\newtheorem*{ratk}{Ratkaisuehdotus}
\newtheorem*{thmextra}{Lisätieto}
\newenvironment{extra}{\begin{thmextra}\footnotesize}{\end{thmextra}}
\newenvironment{serif}{\fontfamily{lmr}\selectfont}{}
\usepackage{ragged2e}   % \justifying
\newenvironment{sample}{\begin{framed}\justifying\begin{serif}}{\end{serif}\end{framed}}



%Matriisien säätö:
\makeatletter
\renewcommand*\env@matrix[1][*\c@MaxMatrixCols c]{%
    \hskip -\arraycolsep
    \let\@ifnextchar\new@ifnextchar
\array{#1}}
\makeatother

\newcommand{\numeroi}{\stepcounter{harkka}\arabic{harkka}}
\newcommand{\vaihto}{\\[1em]}
\newcommand{\R}{\mathbb{R}\,}
\newcommand{\N}{\mathbb{N}\,}
\newcommand{\Z}{\mathbb{Z}\,}
\newcommand{\Q}{\mathbb{Q}\,}
\providecommand{\C}{}               % XeTeX:illä tämä on jokin aksenttikomento
\renewcommand{\C}{\mathbb{C}\,}
\newcommand{\abs}[1]{\left|#1\right|}
\newcommand{\va}{\bar{a}}
\newcommand{\vb}{\bar{b}}
\newcommand{\vw}{\bar{w}}
\newcommand{\vv}{\bar{v}}
%\newcommand{\abs}[1]{\left|#1\right|}
%\newcommand{\BibTeX}{\textsc{Bib}\negthinspace\TeX}
\newcommand{\BibTeX}{Bib\negthinspace\TeX}  % textsc toimii lmodernilla, mutta näyttää pahalta...
\newcommand{\TikZ}{Ti\textit{k}Z}
%\newcommand{\bino}[2]{(x-#1)^{#2}}
\newcommand{\bino}[3]{\binom{#1}{#3}{#2}^{#3}\left(1-#2\right)^{#1-#3}}
\setbeamertemplate{theorems}[numbered]
\newcommand{\set}[1]{ \{ #1 \} }

\title{\LaTeX}



\begin{document}
\begin{frame}
    \titlepage
\end{frame}
\begin{frame}
    \frametitle{Kurssisuunnitelma}
    \tableofcontents
\end{frame}

\section{1. kerta}



%\begin{itemize}
%\item 1. kerta
%\begin{itemize}
%\item Mikä \LaTeX\ on?
%\item Dokumentin rakenne ja luominen
%\item Syntaksi eli ''kielioppi''
%\item Virheilmoitukset
%\item Tekstin muokkaaminen
%\item Matematiikkatila
%\item Numeroidut kaavat
%\end{itemize}
%\item 2. kerta
%\begin{itemize}
%\item Lauseympäristöt
%\item Otsikot
%\item Lauseympäristöjen numerointi
%\item Listat
%\item Omat komennot
%\end{itemize}
%\item 3. kerta
%\begin{itemize}
%\item Taulukot
%\item Kuvien liittäminen
%\item Kuvien piirtäminen
%\item Pakettien hakeminen
%\end{itemize}
%\item 4. kerta
%\begin{itemize}
%\item Sisäiset viittaukset
%\item Lähdeluettelo
%\item Sisällysluettelo ja hakemisto
%\item Kansilehti
%\item Tyylittely
%\end{itemize}
%\end{itemize}

\subsection{Intro}
\begin{frame}
    \frametitle{Mikä \LaTeX\  on?}
    \begin{itemize}[<+->]
        \item Tekstin ladontajärjestelmä
        \item Matemaattisen tekstin tuottamista varten
        \item Kehitetty \TeX -kielen päälle (Leslie Lamport)
        \item Laajalti käytössä ympäri maailman
        \item Merkittävästi erilainen kuin WYSIWYG-järjestelmät
            %ESIMERKKINÄ LAITOKSEN TUTKIELMAPOHJA
    \end{itemize}
\end{frame}
\subsection{Dokumentin rakenne}
\begin{frame}[fragile]
    \frametitle{Dokumentin rakenne ja luominen}
    \LaTeX-dokumentit tuotetaan tällä kurssilla helppokäyttöisellä Texmaker-ohjelmalla.
    \pause
    \vaihto
    Texmakeriin kirjoitetaan dokumentin koodi ja lopuksi tiedosto \textit{käännetään} eli \textit{ajetaan} varsinaiseksi (PDF- tai DVI-) tiedostoksi. Kääntämisen hoitaa \LaTeX\ käyttäjältä piilossa.
    \pause
    \vaihto
    Tarvittavat ohjelmat omalle koneelleen löytää esimerkiksi Matematiikan laitoksen sivujen kautta osoitteesta
    \url{http://wiki.helsinki.fi/pages/viewpage.action?pageId=62428926}
    \pause
    \vaihto
    Kaikkein helpoimmin kuitenkin pääsee alkuun käyttämällä jotain online-ympäristöä kuten esimerkiksi Overleaf (\url{https://www.overleaf.com/}). Tällöin mitään ohjelmia ei tarvitse asentaa eikä tiedostoja siirrellä koneiden välillä.
\end{frame}
\begin{frame}[fragile]
        \begin{harj}
        \begin{itemize}
            \item Avaa Texmaker ja luo uusi tiedosto
            \item Tallenna se muodossa \verb-nimi.tex- johonkin tätä kurssia varten luomaasi kansioon
            \item Kirjoita tiedostoon seuraavat rivit:
                \begin{verbatim}
\documentclass{article}
\begin{document}
Huhuu!
\end{document}
                \end{verbatim}
            \item Aja tiedosto PDFLaTeX:illa
            \item Valitse View PDF
        \end{itemize}
    \end{harj}

    Lopputuloksena tulisi olla PDF-tiedosto, jonka yläreunassa lukee ''Huhuu!''.
\end{frame}
\begin{frame}[fragile]
    \frametitle{Dokumentin rakenne ja luominen}
    \LaTeX-dokumentti koostuu kahdesta osasta: \textit{esittelyosasta}, joka sisältää tarpeellisia asetuksia ja \textit{sisällöstä} eli varsinaisesta dokumentista.
    \pause
    \begin{itemize}[<+->]
        \item Tiedosto ja samalla esittelyosa aloitetaan komennolla \verb+\documentclass{...}+ , jolla valitaan dokumenttiluokka (article)
        \item Esittelyosaan lisätään komentoja tarpeen mukaan
        \item Komento \verb+\begin{document}+ aloittaa itse dokumentin ja lopettaa esittelyosan
        \item Komento \verb+\end{document}+  lopettaa dokumentin eikä sen perään kirjoitettuja rivejä käännetä.
        \item Varsinainen työn sisältö kirjoitetaan siis komentojen \verb+\begin{document}+  ja  \verb+\end{document}+  väliin.
    \end{itemize}
    (Vrt. edellinen harjoitus!)
\end{frame}
\begin{frame}[fragile]
    \frametitle{Dokumentin rakenne ja luominen}
    Esittelyosassa eli heti komennon \verb+\documentclass{}+  jälkeen valitaan käytettävät paketit ja asetukset. \pause Ääkkösiä ja suomalaista tavutusta varten otetaan käyttöön tietyt inputenc-, fontenc- ja babel-paketit:\pause
        \begin{harj}
        Tee dokumenttiisi seuraavat muutokset (älä siis luo uutta tiedostoa):
        %\usepackage[ansinew]{inputenc}
        \begin{Verbatim}[frame=single]
\documentclass[a4paper]{article}
\usepackage[utf8]{inputenc}
\usepackage[T1]{fontenc}
\usepackage[finnish]{babel}
\usepackage{geometry}
\begin{document}
Öö häh? Herätys!
\end{document}
        \end{Verbatim}
    \end{harj}

    Onnistuuko tiedoston ajaminen? Toimivatko ääkköset?
\end{frame}
\begin{frame}[fragile]
    \begin{extra}
        \begin{itemize}
            \item Käyttämällä \verb-inputenc--pakettia \LaTeX{} lukee \verb-tex--tiedoston \verb-utf8--koodattuna, jolloin esim. ä voidaan kirjoittaa tiedostoon sellaisenaan eikä muodossa \verb-\"a-. Tämä vaatii, että käyttämäsi editori tallentaa tiedoston \verb-utf8--koodattuna. Esim. TeXmaker ja TeXstudio tekevät niin oletuksena. Jos ääkköset eivät toimi oikein, niin ongelma on todennäköisesti siinä, että tiedosto on koodattu eri tavalla kuin sitä yritetään lukea. Jos ääkköset eivät toimi, voit yrittää kirjoittaa \verb-[utf8]- sijaan esim. \verb-[latin1]-, \verb-[ansinew]- tai \verb-[applemac]-. Paras olisi kuitenkin käyttää \verb-[utf8]-:aa ja yrittää saada editori tallentamaan tässä muodossa.
            \item Paketti \verb-fontenc- ottaa käyttöön laajemmat fontit, jolloin esim. ä tulostetaan yhtenä merkkinä eikä niin, että merkit a ja \"{} kirjoitetaan päällekkäin. Tällöin vaikkapa tavutus ja pdf-haku saadaan toimimaan oikein.
            \item Lopuksi \verb-babel- asettaa dokumentin kielen suomeksi ottamalla käyttöön suomalaiset tavutussäännöt ym.
        \end{itemize}
    \end{extra}
\end{frame}
\begin{frame}[fragile]
        \begin{harj}
        Kopioi työhösi pari sivullista suomenkielistä tekstiä esimerkiksi Wikipediasta (vältä erikoismerkkejä). Aja tiedosto. Varmista vielä ääkkösten ja tavutuksen toimiminen. Miten kappalejaon kanssa käy?
    \end{harj}

\end{frame}
\begin{frame}[fragile]
    Edellä mainitut paketit ovat esimerkkejä \textit{makropaketeista}, joilla \LaTeX in eli latojan toimintaan voi vaikuttaa. 
    \pause
    Matemaattista tekstiä varten on vielä syytä ottaa käyttöön muutama lisäpaketti. 
        \begin{harj}
        Ota käyttöön paketit\footnote{ams tulee sanoista American Mathematical Society}
        \begin{center}
            \cns{amsmath},\quad \cns{amsthm}\quad ja\quad \cns{amssymb}
        \end{center}
        lisäämällä dokumenttisi esittelyosaan seuraavat komennot:
        \begin{lstlisting}
\usepackage{amsmath}
\usepackage{amsthm}
\usepackage{amssymb}<>
        \end{lstlisting}
        \pause
        Nämä sisältävät fontteja, symboleita ja muuta matemaattisen tekstin kirjoittamiselle tarpeellista. Paketti \cns{amsmath} on ladattava ennen \cns{amsthm}-pakettia.
    \end{harj}

    \pause
    Nämä sisältävät fontteja, symboleita ja muuta matemaattisen tekstin kirjoittamiselle tarpeellista. Paketti \verb-amsmath- on ladattava ennen \verb-amsthm--pakettia.
\end{frame}
\begin{frame}[fragile]
    \frametitle{Syntaksi eli ''kielioppi''}
    \LaTeX in toimintaa ohjataan komennoilla. \pause Niillä tuotetaan esimerkiksi matemaattisia symboleita, korostetaan tekstin osia, luodaan otsikoita, piirretään kuvia, määritellään asetuksia jne.  \pause
    \begin{framed}
        Komennot alkavat aina kenoviivalla \verb-\-
    \end{framed}
    \pause
    Komentoja voi tarvittaessa etsiä esimerkiksi oppaista \begin{scriptsize}
        \url{http://www.ntg.nl/doc/hellgren/lyhyt2e.pdf}
    \end{scriptsize},
    \begin{scriptsize}
        \url{http://en.wikibooks.org/wiki/LaTeX}
    \end{scriptsize} ja
    \begin{scriptsize}
        \url{http://www.rri.res.in/~sanjib/latex/ltx-2.html}
    \end{scriptsize}
\end{frame}
\subsection{Syntaksi}
\begin{frame}[fragile]
    \frametitle{Syntaksi}
    %Edellä on jo tutustuttu komentoihin \verb-\documentclass{}-, \verb-\begin{}...\end{}-, \verb-\usepackage[]{}- jne. Ne ovat esimerkkejä komennoista, jotka
    Komennot tarvitsevat usein \textit{argumentin} (lisämääreen). \pause Se kirjoitetaan komennon perään
    \begin{itemize}[<+->]
        \item aaltosulkuihin \verb-{}-, kun argumentti on pakollinen
        \item hakasulkuihin \verb-[]-, kun argumentti on valinnainen
            %\item Pakollinen argumentti on komennon perässä aaltosuluissa \verb-{}-
            %\item Valinnainen argumentti on komennon perässä hakasuluissa \verb-[]-
    \end{itemize}
    \pause
    Komennolla voi olla yksi tai useampi pakollinen argumentti ja lisäksi yksi tai useampi valinnainen argumentti. \pause Toisilla komennoilla ei ole ainuttakaan argumenttia.
\end{frame}
\begin{frame}[fragile]
    \frametitle{Syntaksi}
    Esimerkiksi 
    \begin{itemize}[<+->]
        \item Komennoilla \verb-\cup- ja \verb-\cap- ei ole yhtään argumenttia (joukkojen yhdiste ja leikkaus)
        \item Komennolla \verb-\sqrt[]{}- on yksi valinnainen ja yksi pakollinen argumentti (valinnainen on juuren kertaluku, pakollinen juurrettava luku. Jos valinnainen argumentti puuttuu, \LaTeX\ tulkitsee neliöjuureksi)
        \item Komennolla \verb-\documentclass[]{}- on myös valinnainen argumentti, jolla voi valita mm. kirjain- ja paperikoon
        \item Komennolla \verb-\frac{}{}- on kaksi pakollista argumenttia (murtoluvun osoittaja ja nimittäjä)
    \end{itemize}
    %Komentoja ja niiden selityksiä voi etsiä esimerkiksi osoitteista
    %\begin{itemize}
    %\item \url{http://www.rri.res.in/~sanjib/latex/ltx-2.html}
    %\end{itemize}
    %PAREMPI LINKKI
    \pause
    Argumenttien järjestyksen ja määrän kanssa on oltava tarkka!
\end{frame}
\begin{frame}[fragile]
    \frametitle{Syntaksi}
    Muotoa 
    \[
    \verb-\begin{ympäristön nimi}-\dots \verb-\end{ympäristön nimi}-.
    \]
    olevilla komentopareilla käytetään ns. ympäristöjä. \pause Näitä voivat olla esimerkiksi lauseet, määritelmät, listat ja taulukot. \pause
    \vaihto Ympäristölläkin voi olla nimen lisäksi muita pakollisia tai valinnaisia argumentteja, kuten tieto taulukon sarakkeiden lukumäärästä tai kuvan toivotusta sijainnista. \pause
    \vaihto
    Kurssin aikana opetellaan käyttämään muutamia tarpeellisia ympäristöjä ja luomaan omia komentoja.
\end{frame}
\begin{frame}[fragile]
    \frametitle{Erikoismerkeistä}
    Jotkin erikoismerkit on varattu \LaTeX in käyttöön:
    \begin{itemize}[<+->]
        \item \verb-%- aloittaa kommenttirivin
        \item \verb-$- aloittaa ja lopettaa tavallisen matematiikkatilan
        \item \verb-\- aloittaa komennon (komento \verb-\\- katkaisee rivin)
        \item \verb-&- on käytössä kun rivejä järjestetään kohdakkain
        \item \verb-{- ja \verb-}- ovat komennon argumentin ympärille tulevat merkit
    \end{itemize}\pause
    Jotta erikoismerkin saisi näkyviin lopullisessa työssä, on käytettävä komentoa:
    \begin{table}[h!]
        \begin{tabular}{ccc}
            Komento & & Tulostus\\
            \hline
            \verb-\%- & & \%\\
            \verb-\$- & & \$\\
%            \verb-\backslash- & & \(\backslash\)\\
            \verb-\textbackslash- & & \textbackslash\\
            \verb-\&- & & \&\\
            \verb-\{\}- & & \{\}
        \end{tabular}
    \end{table}
\end{frame}
\begin{frame}[fragile]
        \begin{harj}
        Kolme peräkkäistä pistettä saa komennolla \lstinline-\dots-. Kokeile, miten lopputulokset eroavat, jos kirjoitat pisteet itse.  Kirjoita sitten seuraava:
        \begin{sample}
            Erikoismerkkejä ovat mm. \%, \$ ja \&. Merkkijonon \textbackslash textbackslash tuottaminen onnistuu näin\dots
        \end{sample}
    \end{harj}

        \begin{harj}
        Kommenttirivin avulla koodin sekaan voi kirjoittaa selkeyttäviä huomautuksia, jotka eivät tule näkyviin lopulliseen työhön. Kommenttirivi aloitetaan merkillä \verb-%- ja päätetään rivinvaihtoon. 
        \vaihto
        Kirjoita koodin sekaan esimerkiksi rivi
        \begin{Verbatim}[frame=single]
%Tämä rivi ei tule näkyviin lopullisessa työssä.
        \end{Verbatim}
        ja testaa, pitääkö väite paikkansa.
    \end{harj}

\end{frame}
\begin{frame}
    \frametitle{Virheilmoitukset}
    Tiedoston ajamisen jälkeen Texmakerin alareunaan saattaa ilmestyä sinisiä ja punaisia viestejä.\pause
    \begin{itemize}[<+->]
        \item Siniset viestit ovat varoituksia ulkonäöllisista seikoista
        \item Punaiset viestit ovat kääntämisen estäviä virheitä
    \end{itemize}\pause
    Punaisten virheilmoitusten kanssa on pakko olla tarkka - virheet tulee korjata heti. 
    \vaihto
    \pause
    Virheilmoituksesta on pääteltävissä jotain sattuneesta virheestä, mutta tämä vaatii totuttelua. 
    \pause Yleensä kyseessä on jonkin komennon väärinkirjoitus, puuttuva aaltosulku tai muu pieni yksityiskohta.
    \vaihto
    \pause Virheilmoituksia saattaa tulla valtavasti vaikka kyse olisi yksittäisestä ongelmasta!
\end{frame}
\subsection{Otsikot ja korostukset}
\begin{frame}[fragile]
    \frametitle{Otsikot}
    Teksti jäsennetään ja otsikoidaan komennoilla 
    \begin{itemize}[<+->]
        \item \verb-\chapter{otsikko}- (vain luokissa book ja report)
        \item \verb-\section{otsikko}- (article-luokan karkein jako)
        \item \verb-\subsection{otsikko}-
        \item \verb-\subsubsection{otsikko}- jne.
    \end{itemize} \pause
    %Dokumenttiluokissa \verb-report- ja \verb-book- on käytössä myös \verb-\chapter-, alilukunaan \verb-\section- jne.
    Otsikot numeroidaan automaattisesti. \pause Numeroinnin saa pois lisäämällä merkin \verb-*- komennon perään, siis esimerkiksi 
    \[
        \verb-\section*{numeroimaton otsikko}-.
    \]
\end{frame}
\begin{frame}
        \begin{harj}
        \begin{itemize}
            \item Jaa tekstisi neljään numeroituun osioon
            \item Jaa ensimmäinen osio lisäksi muutamaksi ali- ja alialiosioksi
            \item Jätä ainakin yksi alialiosio numeroimatta
            \item Nimeä kaikki osiot
        \end{itemize}
        Tee ensimmäisen kerran harjoitukset ensimmäiseen osioon, toisen kerran harjoitukset toiseen osioon jne. 
    \end{harj}

    Tee ensimmäisen kerran harjoitukset ensimmäiseen osioon, toisen kerran harjoitukset toiseen osioon jne. 
\end{frame}
\begin{frame}[fragile]
    \frametitle{Tekstin muokkaaminen}
    Tekstieditorissa eli Texmaker-ohjelmassa käytetty fontti tai tekstin suuruus eivät vaikuta lopulliseen työhön, vaan kaikki ulkoasulliset muutokset on tehtävä komennoilla. (Isot/pienet kirjaimet toimivat kuitenkin sellaisenaan.)
    \vaihto\pause
    Erityisesti kannattaa huomata, että tekstiä kirjoittaessa
    \begin{itemize}[<+->]
        \item peräkkäiset välilyönnit tulostuvat yhdeksi välilyönniksi
        \item myös yksittäinen rivinvaihto tulkitaan välilyönniksi
        \item tyhjä rivi (yksi tai useampi) aloittaa uuden kappaleen
    \end{itemize}
    \pause
    Erikokoisia välilyöntejä varten on omia komentojaan, kuten esimerkiksi \verb-\,-, \verb-\quad- ja \verb-\qquad-.
    \vaihto\pause Pystysuunnassa tyhjää tilaa saa komennolla \verb-\vspace{mitta}-, jossa mitta annetaan esim. pikseleinä (pt) tai senttimetreinä (cm). 
\end{frame}
%\begin{frame}[fragile]
%\frametitle{Tekstin muokkaaminen}
%Koodin sekaan voi kirjoittaa ns. kommenttirivejä, joita ei huomioida käännettäessä eikä siis näytetä lopullisessa työssä. Kommentti alkaa merkillä \verb-%- ja päättyy rivinvaihtoon. Kommentointi kannattaa!
%\vaihto
%Tekstin tasauksen voi valita komennoilla \verb-\centering-, \verb-\flushleft-, \verb-\flushright-. 
%\vaihto
%
%\end{frame}
\begin{frame}[fragile]
    \frametitle{Tekstin muokkaaminen}
    Kirjainkokoa voi kesken tekstin muuttaa lukuisilla komennoilla kuten \verb-\huge-, \verb-\tiny-, ja \verb-\normalsize-, jotka vaikuttavat kunnes kokoa taas muutetaan. 
    \vaihto\pause
    Tekstin \textbf{lihavointi} onnistuu manuaalisesti komennolla \verb-\textbf{lihavoitava teksti}-, \textit{kursivointi} komennolla \verb-\textit{kursivoitava teksti}- ja \underline{alleviivaus} komennolla \verb-\underline{alleviivattava teksti}-. Alleviivaus rikkoo tavutuksen ym., joten ahkeran alleviivaajan kannattaa tutustua esim. \verb-ulem--pakettiin.
    \vaihto\pause
    Sanojen korostamiseen kannattaa kuitenkin käyttää komentoa \verb-\emph{korostettava teksti}-, joka kursivoi tai lihavoi tarpeen mukaan!
\end{frame}
\begin{frame}[fragile]
    \frametitle{Tekstin muokkaaminen}
    Tekstin keskittäminen onnistuu ympäristön \verb-center- avulla:
    \[
    \verb-\begin{center}...keskitetty teksti...\end{center}-
    \]
    Tällä tavoin keskitetyn tekstin ylä- ja alapuolelle jää hieman tyhjää tilaa. \vaihto\pause
    Oikeaan tai vasempaan laitaan tasattua tekstiä saa halutessaan ympäristöillä \verb-flushright- ja \verb-flushleft-.
\end{frame}
\begin{frame}
        \begin{harj}
        Valitse dokumentistasi muutama rivi tekstiä ja keskitä se tai tasaa oikeaan laitaan. Muokkaa sitten tasattua tekstiä käyttämällä erilaisia kirjainkokoja ja korostuksia. Esimerkiksi siis jotain seuraavanlaista:
        \begin{sample}
            \begin{flushright}
                Tämä teksti on normaalikokoista, \tiny mutta tämä pientä ja \huge tämä suurta \normalsize. Tähän laitan suuren välin: \qquad, tähän vähän pienemmän: \quad ja \emph{tätä tekstiä taas haluan korostaa!}
            \end{flushright}
        \end{sample}
    \end{harj}

\end{frame}
\subsection{Matemaattinen teksti}
\begin{frame}[fragile]
    \frametitle{Matematiikkatila}
    \pause
    Matemaattisia ilmaisuja saa tekstin sekaan käyttämällä komentoja \verb+\(+  
    ja \verb+\)+. \pause Esimerkiksi rivi
    \begin{Verbatim}[frame=single]
Yhtälöt \(x^2+y^2-2=0\) ja \(y=2x+1\) toteutuvat 
samanaikaisesti tasan kahdessa tason pisteessä.
    \end{Verbatim}
    \pause
    tulostuu riviksi 
    \begin{sample}
        Yhtälöt \(x^2+y^2-2=0\) ja \(y=2x+1\) toteutuvat samanaikaisesti tasan kahdessa tason pisteessä.
    \end{sample} 
    \pause
    Huomaa, että matematiikkatilassa kirjaimet tulostuvat erilailla, kuin muussa tekstissä. 
    %Huomaa myös, että 
    %\[
    %\verbö\(2x+3=4\)ö
    %\]
    %tuottaa yhtälön \(2x+3=4\) \pause ja komento 
    %\[
    %\verbö\(x^3+\sqrt{2}x^2+1=0\)ö
    %\]
    %yhtälön \(x^3+\sqrt{2}x^2+1=0\). \pause Tätä kutsutaan matematiikkatilaksi.
\end{frame}
\begin{frame}[fragile]
        \begin{harj}
        Tuota seuraava lause dokumenttiisi:
        \begin{sample}
            Vaihdannaisessa renkaassa pätee \((a+b)^2=a^2+2ab+b^2\). Jos siis \(2ab\neq0\), niin \((a+b)^2\neq a^2+b^2\). 
        \end{sample}
        Erisuuruuden saat komennolla \verb-\neq-.
    \end{harj}

\end{frame}
\begin{frame}[fragile]
    \frametitle{Kaavarivi}
    Kun matemaattinen ilmaisu halutaan yksinkertaisesti omalle kaavarivilleen, se kirjoitetaan merkkien \verb-\[- ja \verb-\]- väliin.
    \pause
    Esimerkiksi 
    \begin{footnotesize}
        \begin{Verbatim}[frame=single]
Toisen asteen yhtälön \(ax^2+bx+c=0\) ratkaisu saadaan kaavasta
\[
    x = \frac{-b\pm\sqrt{b^2-4ac}}{2a}.
\]
        \end{Verbatim} 
    \end{footnotesize}
    tuottaa seuraavanlaisen esityksen:
    \pause
    \begin{sample}
        Toisen asteen yhtälön \(ax^2+bx+c=0\)
        ratkaisu saadaan kaavasta
        \[
            x = \frac{-b\pm\sqrt{b^2-4ac}}{2a}.
        \]
    \end{sample}
\end{frame}
\begin{frame}[fragile]
        \begin{harj}\label{rajaArvo}
        Tuota seuraava dokumenttiisi:
        \begin{sample}
            Funktion \(f\colon \R\to \R\) derivaatta pisteessä \(x_0\in\R\) on
            \[
                f'(x_0)=\lim_{x\to x_0}\frac{f(x)-f(x_0)}{x-x_0},
            \]
            mikäli raja-arvo on olemassa.
        \end{sample}
    \end{harj}

    \begin{itemize}
        \item Kaksoispisteen paikalla kannattaa käyttää komentoa \verb-\colon-
        \item Reaalilukujen joukon symbolin saat komennolla \verb-\mathbb{R}-
        \item Tähän yhteyteen oikeanlaisen nuolen saat komennolla \verb-\to-
        \item Relaatio \(\in\) tulostuu komennolla \verb-\in-
        \item Raja-arvo-operaattorin saat komennolla \verb-\lim_{alaindeksi}-.
            %\item Osamäärän saat komennolla \verb-\frac{osoittaja}{nimittäjä}-
    \end{itemize}
\end{frame}
\begin{frame}[fragile]
        \begin{harj}
        Tuota seuraava dokumenttiisi:
        \begin{sample}
            Jos \(F\) on \(\sigma\)-algebra ja \(A_i\in F\) kaikilla \(i= 1,2,\dots\), niin 
            \[
                \bigcup_{i=1}^{\infty} A_i\in F.
            \]
        \end{sample}
        Tarvitset mm. komentoja \lstinline-\sigma-, \lstinline-\in-, \lstinline-\bigcup- ja \lstinline-\infty-. Ala- ja yläindeksit kirjoitetaan merkkien \lstinline-_- ja \lstinline-^- avulla (jos indeksiin halutaan enemmän kuin yksi merkki, se täytyy laittaa aaltosulkeisiin kuten edellisessä tehtävässä). 
    \end{harj}

\end{frame}
\begin{frame}[fragile]
    Yleisimpiä matemaattisia symboleita löytää Texmakerin vasempaan laitaan avautuvista valikoista. Muuten niitä voi etsiä esim. seuraavista osoitteista:
    \begin{scriptsize}
        \begin{itemize}
            \item \url{http://www.tex.ac.uk/tex-archive/info/symbols/comprehensive/symbols-a4.pdf}
            \item \url{http://detexify.kirelabs.org/classify.html}
        \end{itemize}
    \end{scriptsize}
        \begin{harj}
        Selvitä, miten voit tuottaa vektorimerkinnät \(\bar{v}\), \(\bar{w}\) ja \(\overline{AB}\). Kirjoita seuraava:
        \begin{sample}
            Vektoreiden \(\bar{v}=\overline{AB}\) ja \(\bar{w}=\overline{CD}\) ristitulo \(\bar{v}\times\bar{w}\) on kohtisuorassa kumpaakin vektoria vastaan. Vektoreiden pistetulo \(\bar{v}\cdot\bar{w}\) on sen sijaan reaaliluku.
        \end{sample}
    \end{harj}


\end{frame}

\section{2. kerta}
\subsection{Matemaattinen teksti}
\begin{frame}[fragile]
    \frametitle{Sulut}
    Varsinkin kaavariville kirjoitettaessa monet symbolit ovat huomattavan kookkaita. Sulut tulostuvat automaattisesti oikean kokoisina kun niille käytetään komentoja pelkkien merkkien \verb-(- ja \verb-)- sijaan. \vaihto
    \begin{tabular}{cc|cc}
        %\multicolumn{2}{c}{Vasen}&\multicolumn{2}{c}{Oikea}\\
        Komento & Tulostus & Komento & Tulostus\\
        \hline
        \verb-\left(- & \(\left(\right.\) & \verb-\right)-& \(\left.\right)\)\\
        \verb-\left[- & \(\left[\right.\) & \verb-\right]-& \(\left.\right]\)\\
        \verb-\left\lbrace- & \(\left\lbrace\right.\) & \verb-\right\rbrace- & \(\left.\right\rbrace\)\\
        \verb-\left\langle- & \(\left\langle\right.\) & \verb-\right\rangle- & \(\left.\right\rangle\)\\
        \verb-\left|- & \(\left|\right.\) & \verb-\right|- & \(\left.\right|\)\\
    \end{tabular}
    \vaihto
    Jokaista \verb-\left--alkuista komentoa täytyy seurata \verb-\right--alkuinen komento, vähintään \verb-\right.-, joka ei tulosta mitään. Samoin \verb-\right--alkuista komentoa on edellettävä \verb-\left--alkuinen komento, vähintään \verb-\left.-. 
\end{frame}
\begin{frame}[fragile]
        \begin{harj}
        Tuota seuraava rivi dokumenttiisi:
        \begin{sample}
            \[
                \left\lbrace 2^n \,\middle|\, n\in\Z\right\rbrace=\left\lbrace \left(\frac{1}{2}\right)^{-n}\,\middle|\, n\in\Z\right\rbrace
            \]
        \end{sample}
    \end{harj}

    Komennolla \verb-\middle|- saat pystyviivan automaattisesti oikean kokoisena ja komennolla \verb-\,- hieman ylimääräistä väliä niiden ympärille. Tehtävästä \ref{rajaArvo} saat apua symboleiden \(\in\) ja \(\Z\) luomiseen.
\end{frame}
\begin{frame}[fragile]
        \begin{harj}
        Kirjoita seuraava:
        \begin{sample}
            Jos \(F\) on \(\sigma\)-algebra ja \(P\colon F\to\R\) todennäköisyys, niin tapahtumille \(A_1,A_2,\dotsc\in F\) pätee
            \[
                P\left(\bigcup_{i=1}^\infty A_i\right) \leq \sum_{i=1}^\infty P(A_i).
            \]
        \end{sample}
        \begin{extra}
            \begin{scriptsize}
                Tulet todennäköisesti joskus törmäämään myös seuraaviin tapoihin kirjoittaa kaavoja: \verb-$kaava$- ja \verb-$$kaava$$-. Ensimmäinen on käytännössä sama kuin \verb-\(kaava\)-, mutta jos toinen \verb-$- unohtuu, niin syntyvät virheilmoitukset voivat olla vaikeaselkoisia. Toinen taas on \TeX -tapa kirjoittaa kaavarivejä. Se toimii usein samalla tavalla kuin \verb-\[kaava\]-, mutta ei aina, eikä sitä pitäisi \LaTeX issa käyttää.
            \end{scriptsize}
        \end{extra}
    \end{harj}

\end{frame}
\begin{frame}[fragile]
    \frametitle{Pitkät kaavat}
    Toisinaan matemaattiset ilmaisut ovat niin pitkiä, etteivät ne mahdu yhdelle riville. Tällöin voidaan käyttää esimerkiksi ympäristöä \verb-align*-, jolla rivit saadaan allekkain seuraavasti:

    \begin{minipage}{4cm}
        \begin{scriptsize}
            \begin{Verbatim}[frame=single]
\begin{align*}
    xyz &= abc\\
        &= bca\\
        &= cab\\
\end{align*}
            \end{Verbatim}
        \end{scriptsize}
    \end{minipage}
    \begin{minipage}{4cm}
        \begin{align*}
            xyz &= abc\\
                       &= bca\\
                              &= cab\\
        \end{align*}
    \end{minipage}

    Huomaa, ettei \verb-align*--ympäristöä tarvitse erikseen sijoittaa matematiikkatilaan.
\end{frame}
\begin{frame}[fragile]
    \frametitle{Pitkät kaavat}
    Ympäristölle \Verb-align*- rivin katkaisukohta kerrotaan komennolla \verb-\\- ja tasauskohta merkillä \verb-&-. Katkaisukohta täytyy löytyä kaikilta, paitsi viimeiseltä riviltä. Sen sijaan tasauskohdan täytyy löytyä jokaiselta riviltä!
    \vaihto
    Pitkille kaavoille voi vaihtoehtoisesti käyttää ympäristöä \verb-multline*-, jolle kerrotaan vain rivien katkaisukohdat. Se asettelee rivit automaattisesti (yleensä vähän epämääräisesti).
    \vaihto
    Myös tähdettömiä ympäristöjä \verb-align- ja \verb-multline- voi käyttää, jolloin jokainen rivi tulee numeroiduksi.
    %\begin{minipage}{3cm}
    %\begin{scriptsize}
    %\begin{Verbatim}[frame=single]
    %\begin{align*}
    %ax^2+bx+c &= 0\\
    %4a^2x^2+4abx+4ac &= 0\\
    %4a^2x^2+4abx &= -4ac\\
    %4a^2x^2+4abx +b^2 &= b^2-4ac\\
    %(2ax+b)^2 &= b^2 -4ac\\
    %2ax+b &= \pm\sqrt{b^2-4ac}\\
    %x &= \frac{-b\pm\sqrt{b^2-4ac}}{2a}
    %\end{align*}
    %\end{Verbatim}
    %\end{scriptsize}
    %\end{minipage}
    %\begin{minipage}{3cm}
    %\begin{align*}
    %ax^2+bx+c &= 0\\
    %4a^2x^2+4abx+4ac &= 0\\
    %4a^2x^2+4abx &= -4ac\\
    %4a^2x^2+4abx+b^2 &= b^2-4ac\\
    %(2ax+b)^2 &= b^2 -4ac\\
    %2ax+b &= \pm\sqrt{b^2-4ac}\\
    %x &= \frac{-b\pm\sqrt{b^2-4ac}}{2a}
    %\end{align*}
    %\end{minipage}
\end{frame}
%\begin{frame}[fragile]
%Esimerkiksi komennolla
%\begin{scriptsize}
%\begin{Verbatim}[frame=single]
%\begin{align*}
%(3\vv-\vw)\cdot(\vv+\vw)
%&=(3\vv-\vw)\cdot\vv+ (3\vv-\vw)\cdot\vw \\
%&=3\vv\cdot\vv-\vw\cdot\vv+ 3\vv\cdot\vw - \vw\cdot\vw\\
%&=3||\vv||^2+2(\vv\cdot\vw)-||w||^2\\
%&=3\cdot 2^2+2\cdot (-1)-3^2=1.
%\end{align*}
%\end{Verbatim}
%\end{scriptsize}
%saa seuraavan yhtälöketjun kirjoitetuksi allekkain, yhtäsuuruusmerkit kohdakkain:
%\begin{align*}
%(3\vv-\vw)\cdot(\vv+\vw)&=(3\vv-\vw)\cdot\vv+ (3\vv-\vw)\cdot\vw \\
%&=3\vv\cdot\vv-\vw\cdot\vv+ 3\vv\cdot\vw - \vw\cdot\vw\\
%&=3||\vv||^2+2(\vv\cdot\vw)-||w||^2\\
%&=3\cdot 2^2+2\cdot (-1)-3^2=1.
%\end{align*}
%
%\end{frame}
\begin{frame}[fragile]
    %Seuraavassa pohja \verb-align*--ympäristön käyttämistä varten:
    %\vaihto
    %\begin{minipage}{4cm}
    %\begin{scriptsize}
    %\begin{Verbatim}[frame=single]
    %\begin{align*}
    %xyz &= abc\\
    %	&= bca\\
    %	&= cab\\
    %\end{align*}
    %\end{Verbatim}
    %\end{scriptsize}
    %\end{minipage}
        \begin{harj}
        Kirjoita muutaman yhtälön ketju ja sijoita yhtälöt allekkain, tasaten haluamastasi kohdasta. Voit esimerkiksi derivoida vaiheittain funktion \(x^3\sin(\cos(x))\) tai keksiä jotkin muut yhtälöt. Yhtälöiden ei tarvitse olla tosia. Komennolla \verb-\sin- saa sinifunktion tulostettua pystyfontilla.
    \end{harj}

        \begin{harj}
        Kopioi osa edellisen tehtävän yhtälöketjua ja kirjoita se numeroituun tai numeroimattomaan \verb-multline--ympäristöön. 
    \end{harj}

\end{frame}
\begin{frame}[fragile]
    \frametitle{Numeroidut kaavat}
Komennolla \verb-\begin{equation}...\end{equation}- saa luotua samanlaisen kaavarivin kuin komennolla \verb-\[...\]-, mutta numeroinnilla varustettuna. \pause Esimerkiksi

    \begin{scriptsize}
        \begin{Verbatim}[frame=single]
Einsteinin yhtälöön
\begin{equation}
    E=mc^2
\end{equation}
viitataan myöhemmin.
        \end{Verbatim}
    \end{scriptsize}

    tuottaa seuraavan rivin:
    \begin{sample}
        Einsteinin yhtälöön
        \begin{equation}
            E=mc^2
        \end{equation}
        viitataan myöhemmin.
    \end{sample}
    \pause
    Numeroituihin kaavoihin viittaamiseen palataan tuonnempana.
\end{frame}
\begin{frame}[fragile]
        \begin{harj}\label{viittausTehtava}
        Tuota seuraava dokumenttiisi:
        \begin{sample}
            Eksponenttifunktiolla \(e^x\) on sarjakehitelmä
            \setcounter{equation}{0}
            \begin{equation}
                e^x=\sum_{n=1}^\infty\frac{x^n}{n!}.
            \end{equation}
        \end{sample}
        Summamerkinnän saat komennolla \verb-\sum_{}^{}-, osamäärän komennolla \verb-\frac{}{}- ja äärettömän symbolin komennolla \verb-\infty-. 
    \end{harj}

\end{frame}
%\section{2. kerta}
\subsection{Lauseympäristö}
\begin{frame}[fragile]
    \frametitle{Lauseympäristö}
    Paketti amsthm tarjoaa mahdollisuuden esittää mm. lauseet, lemmat ja määritelmät tyylikkäästi. 
    \vaihto
    Tarvittavat ympäristöt määritellään esittelyosassa komennolla \verb-\newtheorem{}{}-. Ensimmäisiin aaltosulkeisiin tulee nimi, jolla ympäristöä käytetään ja jälkimmäisiin ympäristön otsikko, joka halutaan näkyväksi lopullisessa työssä.
    \vaihto
Komento \verb-\newtheorem{esim}{Esimerkki}- loisi ympäristön, jota käytettäisiin komennolla \verb-\begin{esim}...\end{esim}- ja jonka otsikko valmiissa työssä olisi Esimerkki. 
    %\begin{Verbatim}[frame=single]
    %\newtheorem{esim}{Esimerkki}
    %\newtheorem{lause}{Lause}
    %\newtheorem{maar}{Määritelmä}
    %\end{Verbatim}
\end{frame}
\begin{frame}[fragile]
        \begin{harj}\label{harjYmparistot}
        Luo esittelyosassa ainakin ympäristöt Lause, Määritelmä ja Esimerkki. Käytä luomiasi ympäristöjä ainakin kerran. 
    \end{harj}

Lauseiden todistuksia varten on oma ympäristönsä, jota käytetään komennoilla \verb-\begin{proof}...\end{proof}-
        \begin{harj}
        Kirjoita edellisessä harjoituksessa luomallesi lauseelle jokin todistus. Todistukseksi kelpaa muutama rivi valitsemaasi tekstiä.
    \end{harj}

    %Esimerkiksi rivi
    %\begin{Verbatim}[frame=single]
    %\begin{esim}
    %Tämä on esimerkki\ldots
    %\end{esim}
    %\end{Verbatim}
    %\pause
    %tuottaisi nyt ympäristön
    %\begin{framed}
    %\begin{esim}
    %Tämä on esimerkki\ldots
    %\end{esim}
    %\end{framed}
    %Jos ympäristöä \verb-esim- ei ole määritelty esittelyosassa, tämä aiheuttaa virheen!
\end{frame}
%\begin{frame}[fragile]
%\pause
%Vastaavasti esimerkiksi komento
%\begin{Verbatim}[frame=single]
%\begin{lause}
%Tämä on lause\ldots
%\end{lause}
%\end{Verbatim}
%tuottaa ympäristön
%\pause
%\begin{framed}
%\begin{lause}
%Tämä on lause\ldots
%\end{lause}
%\end{framed}
%\pause
%Kokeile, miltä nämä näyttävät omassa työssäsi.
%\end{frame}
\begin{frame}[fragile]
    \frametitle{Lauseympäristön tyyli}
    Lauseympäristön tyyli valitaan esittelyosassa komennolla \verb-\theoremstyle{...}-. Valittavissa on tyylit \verb-plain-, \verb-definition- ja \verb-remark-. Tyylin valinta vaikuttaa sitä seuraaviin komennolla \verb-\newtheorem{}{}- luotuihin ympäristöihin.
    \vaihto
    Esimerkiksi kirjoittamalla
    \pause
    \begin{Verbatim}[frame=single]
\theoremstyle{plain}
\newtheorem{lemma}{Lemma}
\newtheorem{lause}{Lause}
\theoremstyle{definition}
\newtheorem{maar}{Määritelmä}
    \end{Verbatim}
    \pause
    ympäristöt \verb-lemma- ja \verb-lause- tulostuvat plain-tyylin mukaisesti ja ympäristö \verb-maar- definition-tyylin mukaisesti.
\end{frame}
\begin{frame}[fragile]
        \begin{harj}
        Valitse harjoituksessa \ref{harjYmparistot} luomillesi lauseympäristöille jotkin tyylit. Kokeile, miltä erilaiset tyylit näyttävät ja valitse mieleisesi!
    \end{harj}

\end{frame}
\begin{frame}[fragile]
    \frametitle{Lauseympäristöjen numerointi}
    Komennolla \verb-\newtheorem{}{}- luotu ympäristö luo samalla uuden \emph{laskurin}, jonka mukaan ympäristön toteutumat numeroidaan. 
    \vaihto
    Laskuri voidaan myös asettaa toiselle laskurille \emph{alisteiseksi} tai valita mielivaltaisesti. Erityisesti eri ympäristöt voivat käyttää samaa laskuria, jos niin halutaan.
    \vaihto
    Komennolla \verb-\newtheorem{}{}[]- on valinnaisena argumenttina laskuri, jolle ympäristön numerointi halutaan alisteiseksi. Tämä voisi olla esim. laskuri \verb-section-, jolloin ympäristö numeroitaisiin muodossa ''osionNumero.ympäristönNumero''.
\end{frame}
\begin{frame}[fragile]
    \frametitle{Lauseympäristöjen numerointi}
    Jos ympäristön halutaan käyttävän jotain tiettyä laskuria, käytetään komentoa \verb-\newtheorem{}[laskurin nimi]{}-. Huomaa, että tässä valinnainen argumentti sijoittuu pakollisten väliin. 
    \vaihto
    Esimerkiksi koodilla
    \begin{Verbatim}[frame=single]
\newtheorem{teor}{Teoreema}[section]
\newtheorem{lemma}[teor]{Lemma}
    \end{Verbatim}
    luodut Lemma- ja Teoreema-ympäristöt noudattavat samaa, laskurille \verb-section- alisteista numerointia.
    %\begin{framed}
    %\setcounter{laskuri}{0}
    %\begin{trma}
    %\ldots
    %\end{trma}
    %\begin{prop}
    %\ldots
    %\end{prop}
    %\end{framed}
\end{frame}
\begin{frame}[fragile]
    Kokonaan numeroimattoman lauseympäristön voi luoda komennolla \verb-\newtheorem*{}{}-.
        \begin{harj}
        Aseta yksi lauseympäristöistäsi (esim. Lause) laskurille \verb-section- alisteiseksi.
        %Anna yhdelle lauseympäristöistäisi numeroinniksi \verb-\section- tai \verb-\subsection-.
        Anna sitten toisen lauseympäristön (esim. Lemma) laskuriksi edellinen lauseympäristö. 
    \end{harj}

        \begin{harj}
        Luo jokin numeroimaton lauseympäristö ja käytä sitä työssäsi.
    \end{harj}

\end{frame}
\subsection{Sisäiset viittaukset}
\begin{frame}[fragile]
    \frametitle{Sisäiset viittaukset}
    \LaTeX illa voi helposti viitata numeroituihin kohteisiin, kuten lauseisiin tai yhtälöihin. Viitattava kohde täytyy ensin nimetä komennolla \verb-\label{nimi}- (nimi ei tulostu työhön). Tämän jälkeen viittaaminen onnistuu komennoilla 
    \begin{itemize}
        \item \verb-\ref{nimi}- (tulostaa viitattavan kohteen numeron)
        \item \verb-\eqref{nimi}- (tulostaa kohteen numeron sulkujen sisällä)
        \item \verb-\pageref{nimi}- (tulostaa sen sivun numeron, jolla kohde on)  
    \end{itemize}
    Sisäisten viittausten kanssa tulee aina käyttää komentoja. Tällöin viittaukset pysyvät kohdallaan vaikka numeroinnit muuttuisivat työn edetessä.
\end{frame}
\begin{frame}[fragile]
    Huom! Uuden viittauksen jälkeen työn joutuu ajamaan kahdesti, jotta numeroinnit tulevat näkyviin (kahden kysymysmerkin sijaan).
    \begin{extra}
        Jos \LaTeX-tiedoston nimi on \verb-nimi.tex-, niin ensimmäisellä ajokerralla \LaTeX\ luo (tai päivittää) tiedoston \verb-nimi.aux-, joka sisältää viittaustiedot (kokeile avata tiedosto). Toisella ajokerralla viittaustiedot luetaan tästä ja päivitetään lopulliseen tiedostoon. Tätä ei voi yhdellä ajokerralla tehdä, koska viittaukset voivat viitata myös tekstissä eteenpäin.
    \end{extra}
        \begin{harj}
        Viittaa harjoituksessa \ref{viittausTehtava} kirjoittamaasi numeroituun yhtälöön. Käytä komentoja \verb-\eqref{}- ja \verb-\pageref{}-.
        \vaihto
        Huomaa, että viitattavalle kohteelle on ensin annettava tunnus komennolla \verb-\label{valitsemasi tunnus}-. Tämä kirjoitetaan ympäristön aloittavan komennon \verb-\begin{}- jälkeen. 
    \end{harj}

\end{frame}
\subsection{Listarakenteet}
\begin{frame}[fragile]
    \frametitle{Listarakenteet}
    \LaTeX illa voi luoda listoja mm. ympäristöjen \verb-itemize- ja \verb-enumerate- avulla. Ensimmäinen on ranskalaiset viivat-tyyppinen, jälkimmäinen numeroi listan jäsenet. Näitä käytetään seuraavasti:
    \vaihto
    \begin{minipage}{4cm}
        \begin{scriptsize}
            \begin{Verbatim}[frame=single]
Tässä listani:
\begin{itemize}
    \item asia
    \item toinen asia
    \item kolmas asia
\end{itemize}
            \end{Verbatim}
        \end{scriptsize}
    \end{minipage}
    \begin{minipage}{4cm}
        Tässä listani:
        \begin{itemize}
            \item asia
            \item toinen asia
            \item kolmas asia
        \end{itemize}
    \end{minipage}
    \vaihto
    Luettelointiin käytetyt symbolit voi tarvittaessa valita vapaasti. 
\end{frame}
\begin{frame}[fragile]
    \frametitle{Listat}
    Seuraavassa tehtävässä harjoitellaan sisäkkäisten listojen käyttöä.
        \begin{harj}
        Luo ainakin kolmen kohdan numeroitu lista haluamistasi asioista. Luo yhdeksi listan jäseneksi toinen lista ja yhdeksi tämän listan jäseneksi kolmas lista. Käytä (ainakin) viimeiseen listaan numeroimatonta \verb-itemize--ympäristöä.
    \end{harj}

        \begin{harj}
    Luo vielä yksi lista, mutta käytä tällä kertaa ympäristöä \verb-description-. Se toimii kuten \verb-itemize-, mutta pelkän komennon \verb-\item- sijaan käytetään komentoa \verb-\item[nimi]-, jossa \verb-nimi- on vapaasti valitsemasi merkkijono.
        %\begin{scriptsize}
        %\begin{Verbatim}[frame=single]
        %\begin{description}
        %  \item[T1] The first item
        %  \item[T2] The second item
        %  \item[T3] The third etc \ldots
        %\end{description}
        %\end{scriptsize}
    \end{harj}

    %Listaukseen käytettäviin symboleihin voi itse vaikuttaa. Opetellaan myöhemmin laskareissa usein käytettävä aakkostettu lista.
\end{frame}
\subsection{Omat komennot}
\begin{frame}[fragile]
    \frametitle{Omat komennot}
    %\LaTeX in käyttämistä helpottaa se, että kaikki on koodattavissa komennoiksi. 
    %Esimerkiksi olisi raskasta kirjoittaa jokainen ristitulon merkki pikseli pikseliltä, joten tälle kannattaa luoda oma komentonsa, jolla asia hoituu automaattisesti.
    Omia komentoja luodaan esittelyosassa komennolla \verb-\newcommand{}{}-. Ensimmäinen argumentti on komennon nimi, toiseksi argumentiksi tulee komennon sisältö. Esimerkiksi
    \begin{Verbatim}[frame=single]
\newcommand{\R}{\mathbb{R}}
    \end{Verbatim}
    luo komennon \verb-\R-, joka tulostaa matematiikkatilassa symbolin \(\R\). Komennon luomisen jälkeen kyseisen symbolin tuottaminen onnistuu helposti.
    \vaihto
    Omia komentoja luomalla voit yksinkertaistaa ja helpottaa omaa työtäsi. Jo parikin kertaa toistuva komentojen ketju kannattaa määritellä esittelyosassa yhdeksi yksinkertaiseksi komennoksi.
\end{frame}
\begin{frame}[fragile]
        \begin{harj}
        Luo komennot merkinnöille \(\N\), \(\Z\), \(\Q\), \(\R\) ja \(\C\). Kirjoita seuraava:
        \begin{sample}
            Lukujoukot muodostavat tornin
            \[
                \N\subset \Z\subset \Q\subset \R\subset \C.
            \]
        \end{sample}
    \end{harj}

    Osajoukkorelaation saat komennolla \verb-\subset-. Kokeile myös, mitä komennot \verb-\subseteq-, \verb-\subsetneq- ja  \verb-\supset- tuottavat. Miten saisit symbolit \(\supsetneq\) ja  \(\supseteq\)? Entä symbolin \(\not\subset\)? Kirjoita työhösi vielä seuraava: \(\Z\not\subset\N\).
\end{frame}

\begin{frame}[fragile]
    \frametitle{Omat komennot}
    Komennolla \verb-\newcommand{}[]{}- on valinnaisena argumenttina luotavan komennon argumenttien lukumäärä. Esimerkiksi
    \verb-\newcommand{\set}[1]{ \left\lbrace #1 \right\rbrace }- loisi komennon \verb-\set{}-, jolla voisi tuottaa joukkomerkinnän. 
    \vaihto
    \begin{minipage}{4.2cm}
        \begin{scriptsize}
            \begin{Verbatim}[frame=single]
Toisinaan 
\(
    \N=\set{0,1,2,3,\dots}
\)
            \end{Verbatim}
        \end{scriptsize}
    \end{minipage}
    \begin{minipage}{6cm}\fontfamily{lmr}\selectfont
        Toisinaan 
        \(
        \N=\set{0,1,2,3,\dots}
        \)
    \end{minipage}
    %
%\verb-\newcommand{\komento}[2]{...}- loisi komennon, jota käytettäisiin muodossa \verb-\komento{}{}-. 
    %
    %
    %
    % Esimerkiksi 
    %\begin{scriptsize}
    %\begin{Verbatim}[frame=single]
    %\newcommand{\bino}[3]{\binom{#1}{#3}{#2}^{#3}\left(1-#2\right)^{#1-#3}}
    %\end{Verbatim}
    %
    %\end{scriptsize}
%luo komennon \verb-\bino-, jolla on kolme argumenttia: komento \verb-\bino{n}{p}{k}- tuottaa binomitodennäköisyyden kaavan \[\bino{n}{p}{k}.\]
\end{frame}
\begin{frame}[fragile]
    \frametitle{Omat komennot}
    Komennon \verb-\newcommand{}{}- käyttäminen aiheuttaa konfliktin, jos samanniminen komento on jo käytössä. Tällöin kannattaa nimetä oma komentonsa toisin. 
    \vaihto
    \begin{extra}
        Jos välttämättä haluaa korvata valmiin komennon omallaan, voi käyttää komentoa \verb-\renewcommand{}{}-. Tämä on joskus tarpeen, mutta saattaa sotkea pahasti asioita!
    \end{extra}
\end{frame}
\begin{frame}[fragile]
        \begin{harj}
        \LaTeX issa ei ole valmista komentoa itseisarvofunktiota varten. Korjaa puute luomalla komento \verb-\abs{}-, jonka argumentti on itseisarvomerkkien sisään tuleva lauseke. Tuota sen avulla seuraavat kolmioepäyhtälöt:
        \begin{sample}
            \[
                \abs{\abs{x}-\abs{y}}\leq \abs{x+y}\leq\abs{x}+\abs{y}
            \]
            ja
            \[
                \abs{\int_a^b f(x)\,dx} \leq\int_a^b\abs{f(x)}\, dx.
            \]
        \end{sample}
        Huomaa, että itseisarvomerkkien koon on syytä muuttua niiden sisältämän lausekkeen koon mukaan.\vaihto Määrätyn integraalin saat komennolla \verb-\int_{}^{}- ja pienen välin komennolla \verb-\,- (ennen merkkiä \(dx\)).
    \end{harj}

\end{frame}

\end{document}
