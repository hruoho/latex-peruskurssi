\documentclass[handout,%
hyperref={unicode,colorlinks=true}]{beamer}
\usecolortheme{seahorse}
\usepackage{etex}   % lisää rekistereitä
%\usefonttheme{serif}
\usefonttheme[onlymath]{serif}     % serif vain matikkamoodiin
\usepackage[utf8]{inputenc}
\usepackage[T1]{fontenc}
\usepackage[finnish]{babel}
\usepackage{amsmath}
%\usepackage{amsfonts}
\usepackage{amssymb}
\usepackage{tikz}
\usepackage{framed}
\usepackage{alltt}
\usepackage{fancyvrb}
\usepackage{multirow}
\usepackage{mathtools}
\usepackage{multicol}
\usetikzlibrary{matrix,arrows,decorations.pathmorphing}
\usetikzlibrary{fadings,patterns,arrows}
\pgfmathsetmacro{\inf}{-2}
\pgfmathsetmacro{\sup}{2}
\tikzstyle{MyPlotStyle}=[domain=\inf:\sup,samples=100,smooth]
\usepgflibrary{arrows}

\usepackage{float}

\usepackage{lmodern}
\usepackage{microtype}
\usepackage{listings}
\lstset{
    language=[LaTeX]TeX,
    basicstyle=\ttfamily,%\footnotesize,
    commentstyle=\ttfamily,%\footnotesize,
    columns=fullflexible,
    keepspaces=true,
    backgroundcolor=\color{gray!20},
    escapechar=<>,
    morekeywords={\part,\chapter,\subsection,\subsubsection,\paragraph,\subparagraph,\middle,\theoremstyle,\eqref,\subsetneq,\usetikzlibrary,\definecolor,\tableofcontents,\maketitle,\multirow},
    literate={ä}{{\"a}}1
             {Ä}{{\"A}}1
             {ö}{{\"o}}1
             {Ö}{{\"O}}1
}
\lstdefinelanguage{BibTeX}
    {keywords={%
        @article,@book,@collectedbook,@conference,@electronic,@ieeetranbstctl,%
        @inbook,@incollectedbook,@incollection,@injournal,@inproceedings,%
        @manual,@mastersthesis,@misc,@patent,@periodical,@phdthesis,@preamble,%
        @proceedings,@standard,@string,@techreport,@unpublished%
    },
    comment=[l][\itshape]{@comment},
    sensitive=false,
}

\newcommand{\cns}[1]{\texttt{#1}}   % vakioille yms.
\newcommand{\newterm}[1]{\textit{#1}}

\newcounter{harkka}
\newtheorem{lause}{Lause}
\newtheorem*{huom}{Huomautus}
\newcounter{laskuri}
\newtheorem{trma}{Teoreema}[laskuri]
\newtheorem{prop}[trma]{Propositio}
\theoremstyle{remark}
\newtheorem{harj}{Harjoitus}[section]

\newenvironment{fframe}
    {\begin{frame}[fragile,environment=fframe]}
    {\end{frame}}

\newtheorem*{esim}{Esimerkki}
\newtheorem*{ratk}{Ratkaisuehdotus}
\newtheorem*{thmextra}{Lisätieto}
\usepackage{ragged2e}   % \justifying
\newenvironment{extra}{\begin{thmextra}\footnotesize\RaggedRight}{\end{thmextra}}
\newenvironment{serif}{\fontfamily{lmr}\selectfont}{}
\newenvironment{sample}{\begin{framed}\justifying\begin{serif}}{\end{serif}\end{framed}}



%Matriisien säätö:
\makeatletter
\renewcommand*\env@matrix[1][*\c@MaxMatrixCols c]{%
    \hskip -\arraycolsep
    \let\@ifnextchar\new@ifnextchar
\array{#1}}
\makeatother

\newcommand{\numeroi}{\stepcounter{harkka}\arabic{harkka}}
\newcommand{\vaihto}{\\[1em]}
\newcommand{\R}{\mathbb{R}\,}
\newcommand{\N}{\mathbb{N}\,}
\newcommand{\Z}{\mathbb{Z}\,}
\newcommand{\Q}{\mathbb{Q}\,}
\providecommand{\C}{}               % XeTeX:illä tämä on jokin aksenttikomento
\renewcommand{\C}{\mathbb{C}\,}
\newcommand{\abs}[1]{\left|#1\right|}
\newcommand{\va}{\bar{a}}
\newcommand{\vb}{\bar{b}}
\newcommand{\vw}{\bar{w}}
\newcommand{\vv}{\bar{v}}
%\newcommand{\abs}[1]{\left|#1\right|}
%\newcommand{\BibTeX}{\textsc{Bib}\negthinspace\TeX}
\newcommand{\BibTeX}{Bib\negthinspace\TeX}  % textsc toimii lmodernilla, mutta näyttää pahalta...
\newcommand{\TikZ}{Ti\textit{k}Z}
%\newcommand{\bino}[2]{(x-#1)^{#2}}
\newcommand{\bino}[3]{\binom{#1}{#3}{#2}^{#3}\left(1-#2\right)^{#1-#3}}
\setbeamertemplate{theorems}[numbered]
\newcommand{\set}[1]{ \{ #1 \} }

\title{\LaTeX}

